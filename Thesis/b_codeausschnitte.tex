\chapter{Codeausschnitte}
\section{Angepasstes Skript Körperteil}
\label{code:körperteil}
\begin{lstlisting}[caption={Körperteil Skript},captionpos=b,label={lst:skript_körperteil}]
using System.Collections.Generic;
using UnityEngine;
using UnityEngine.Events;

[System.Serializable]
public class PhysicsConfig
{
    public float maxJointSpring = 40000f;
    public float jointDampen = 5000f;
    public float maxJointForceLimit = 20000f;
    public float k_MaxAngularVelocity = 50f;
}

/*
* Bodypart
* implementes functions to controll the joints and holds information about bodypart
*/
[RequireComponent(typeof(Rigidbody))]
public class Bodypart : MonoBehaviour
{
    [Header("Body Part Config")]
    public PhysicsConfig physicsConfig;
    public bool triggerTouchingGroundEvent;

    [HideInInspector] public UnityEvent onTouchedGround;
    [HideInInspector] public bool touchingGround;

    [HideInInspector] public UnityEvent onTouchedTarget;

    //component references
    [HideInInspector] public Rigidbody rb;
    [HideInInspector] public ConfigurableJoint joint;

    //starting pos and rot
    [HideInInspector] public Vector3 startingPos;
    [HideInInspector] public Quaternion startingRot;

    //bodypart information
    [HideInInspector] public Vector3 dof; //degrees of freedom
    [HideInInspector] public float currentStrength;
    [HideInInspector] public float power;

    void Awake()
    {
        //get component references
        rb = GetComponent<Rigidbody>();
        if (TryGetComponent(out ConfigurableJoint foundJoint))
        {
            joint = foundJoint;
        }

        //save starting pos and rot
        startingPos = transform.position;
        startingRot = transform.rotation;

        //setup rigidbody max angular velocity
        rb.maxAngularVelocity = physicsConfig.k_MaxAngularVelocity;

        //setup base degrees of freedom
        dof = new Vector3(0f, 0f, 0f);

        if (joint)
        {
            //set joint settings from physics config
            JointDrive jd = new JointDrive
            {
                positionSpring = physicsConfig.maxJointSpring,
                positionDamper = physicsConfig.jointDampen,
                maximumForce = physicsConfig.maxJointForceLimit
            };
            joint.slerpDrive = jd;

            //calculate degrees of freedom
            dof.x = joint.angularXMotion != ConfigurableJointMotion.Locked ? 1 : 0;
            dof.y = joint.angularYMotion != ConfigurableJointMotion.Locked ? 1 : 0;
            dof.z = joint.angularZMotion != ConfigurableJointMotion.Locked ? 1 : 0;
        }
    }

    public void Initialize()
    {
        //setup events
        onTouchedGround = new UnityEvent();
        onTouchedTarget = new UnityEvent();
    }

    /// <summary>
    /// Reset body part to initial configuration.
    /// </summary>
    public void ResetTransform()
    {
        rb.transform.position = startingPos;
        rb.transform.rotation = startingRot;
        rb.velocity = Vector3.zero;
        rb.angularVelocity = Vector3.zero;
        touchingGround = false;
    }

    public void ResetTransform(Vector3 position, Quaternion rotation)
    {
        rb.transform.position = position;
        rb.transform.rotation = rotation;
        rb.velocity = Vector3.zero;
        rb.angularVelocity = Vector3.zero;
        touchingGround = false;
    }

    /// <summary>
    /// Apply torque according to defined goal `x, y, z` angle and force `strength`.
    /// </summary>
    public void SetJointTargetRotation(float x, float y, float z)
    {
        x = (x + 1f) * 0.5f;
        y = (y + 1f) * 0.5f;
        z = (z + 1f) * 0.5f;

        var xRot = Mathf.Lerp(joint.lowAngularXLimit.limit, joint.highAngularXLimit.limit, x);
        var yRot = Mathf.Lerp(-joint.angularYLimit.limit, joint.angularYLimit.limit, y);
        var zRot = Mathf.Lerp(-joint.angularZLimit.limit, joint.angularZLimit.limit, z);

        joint.targetRotation = Quaternion.Euler(xRot, yRot, zRot);
    }

    public void SetJointStrength(float strength)
    {
        var rawVal = (strength + 1f) * 0.5f * physicsConfig.maxJointForceLimit;
        var jd = new JointDrive
        {
            positionSpring = physicsConfig.maxJointSpring,
            positionDamper = physicsConfig.jointDampen,
            maximumForce = rawVal
        };
        joint.slerpDrive = jd;
        currentStrength = jd.maximumForce;
    }

    void OnCollisionEnter(Collision col)
    {
        if (col.transform.CompareTag("ground"))
        {
            touchingGround = true;
            if (triggerTouchingGroundEvent)
            {
                onTouchedGround.Invoke();
            }
        }
        if (col.transform.CompareTag("target"))
        {
            onTouchedTarget.Invoke();
        }
    }

    void OnCollisionExit(Collision other)
    {
        if (other.transform.CompareTag("ground"))
        {
            touchingGround = false;
        }
    }

    void FixedUpdate()
    {
        //calculate joint power
        if (joint)
        {
            Vector3 currentTorque = joint.currentTorque;
            Vector3 angularVel = rb.angularVelocity;

            power = Mathf.Abs(Vector3.Dot(currentTorque, angularVel));
        }
    }
}
\end{lstlisting}

\newpage
\section{Angepasstes Skript Agenten}
\label{code:agent}
\begin{lstlisting}[caption={Agenten Skript},captionpos=b,label={lst:skript_agent1}]
using System;
using System.Collections.Generic;
using UnityEngine;
using UnityEngine.Events;
using Unity.MLAgents;
using Unity.MLAgents.Actuators;
using Unity.MLAgents.Sensors;
using Random = UnityEngine.Random;

/*
* Walker Agent 1
* brain of the walker controlling the bodyparts and implementing the rl loop
*/
public class WalkerAgent1 : Agent
{
    [Header("Walk Speed")]
    public float minWalkingSpeed = 0.1f;
    public float maxWalkingSpeed = 10f;
    public bool randomizeWalkSpeedEachEpisode;
    public float targetWalkingSpeed = 0.1f;


    [Header("Target To Walk Towards")]
    public Transform target;
    public UnityEvent onTouchedTarget = new UnityEvent();

    [Header("Bodyparts")]
    public Transform root;
    public Transform head;
    public bool randomizeRotationOnEpsiode = true;
    [HideInInspector] public List<Bodypart> bodyparts = new List<Bodypart>();

    [Header("Debug Log Stats")]
    public bool logStats = false;

    //This will be used as a stabilized model space reference point for observations
    //Because ragdolls can move erratically during training, using a stabilized reference transform improves learning
    protected OrientationCubeController1 walkOrientationCube;
    public EnvironmentParameters m_ResetParams;
    public StatsRecorder statsRecorder;

    /*
    * Environment stats
    */
    protected Vector3 previousPos;
    protected float distanceMovedInTargetDirection;
    protected float lastReachedTargetTime = 0f;
    protected int reachedTargets;

    public override void Initialize()
    {
        //init orientation object
        GameObject orientationObject = new GameObject("OrientationObject");
        orientationObject.transform.parent = transform;
        walkOrientationCube = orientationObject
        .AddComponent<OrientationCubeController1>();
        walkOrientationCube.root = root;
        walkOrientationCube.target = target;

        //change to auto setup each body part
        foreach (Bodypart bp in root.GetComponentsInChildren<Bodypart>())
        {
            bp.Initialize();
            bp.onTouchedGround.AddListener(OnTouchedGround);
            bp.onTouchedTarget.AddListener(OnTouchedTarget);
            bodyparts.Add(bp);
        }

        m_ResetParams = Academy.Instance.EnvironmentParameters;
        statsRecorder = Academy.Instance.StatsRecorder;
    }

    /// <summary>
    /// Loop over body parts and reset them to initial conditions.
    /// </summary>
    public override void OnEpisodeBegin()
    {
        //Reset all of the body parts
        foreach (Bodypart bp in bodyparts)
        {
            bp.ResetTransform();
        }

        //Random start rotation to help generalize
        if (randomizeRotationOnEpsiode)
        {
            root.rotation = Quaternion.Euler(0, Random.Range(0.0f, 360.0f), 0);
        }

        //Set our goal walking speed
        targetWalkingSpeed =
            randomizeWalkSpeedEachEpisode ? Random.Range(minWalkingSpeed, maxWalkingSpeed) : targetWalkingSpeed;

        //record walking speed stats
        RecordStat("Environment/WalkingSpeed", targetWalkingSpeed);

        //record then reset distance moved in target direction
        RecordStat("Environment/DistanceMovedInTargetDirection", distanceMovedInTargetDirection);
        distanceMovedInTargetDirection = 0f;
        previousPos = root.position;

        //record then reset targets reached
        RecordStat("Environment/ReachedTargets", reachedTargets);
        reachedTargets = 0;
    }

    /// <summary>
    /// Add relevant information on each body part to observations.
    /// </summary>
    public void CollectObservationBodyPart(Bodypart bp, VectorSensor sensor)
    {
        //GROUND CHECK
        sensor.AddObservation(bp.touchingGround); // Is this bp touching the ground

        //Get velocities in the context of our orientation cube's space
        //Note: You can get these velocities in world space as well but it may not train as well.
        sensor.AddObservation(walkOrientationCube.transform
        .InverseTransformDirection(bp.rb.velocity));
        sensor.AddObservation(walkOrientationCube.transform
        .InverseTransformDirection(bp.rb.angularVelocity));

        //Get position relative to root in the context of our orientation cube's space
        sensor.AddObservation(walkOrientationCube.transform
        .InverseTransformDirection(bp.rb.position - root.position));

        //return when bodypart joint has no free rotation axis
        if (bp.dof.sqrMagnitude <= 0) return;

        sensor.AddObservation(bp.rb.transform.localRotation);
        sensor.AddObservation(bp.currentStrength / bp.physicsConfig.maxJointForceLimit);
    }

    /// <summary>
    /// Loop over body parts to add them to observation.
    /// </summary>
    public override void CollectObservations(VectorSensor sensor)
    {
        var cubeForward = walkOrientationCube.transform.forward;

        //velocity we want to match
        var velGoal = cubeForward * targetWalkingSpeed;
        //ragdoll's avg vel
        var avgVel = GetAvgVelocity();

        //current ragdoll velocity. normalized
        sensor.AddObservation(Vector3.Distance(velGoal, avgVel));
        //avg body vel relative to cube
        sensor.AddObservation(walkOrientationCube.transform
        .InverseTransformDirection(avgVel));
        //vel goal relative to cube
        sensor.AddObservation(walkOrientationCube.transform
        .InverseTransformDirection(velGoal));

        //rotation deltas
        sensor.AddObservation(Quaternion.FromToRotation(root.forward, cubeForward));
        sensor.AddObservation(Quaternion.FromToRotation(head.forward, cubeForward));

        //Position of target position relative to cube
        sensor.AddObservation(walkOrientationCube.transform
        .InverseTransformPoint(target.transform.position));

        foreach (Bodypart bp in bodyparts)
        {
            CollectObservationBodyPart(bp, sensor);
        }
    }

    public override void OnActionReceived(ActionBuffers actionBuffers)
    {
        var continuousActions = actionBuffers.ContinuousActions;
        int i = -1;

        foreach (Bodypart bp in bodyparts)
        {
            if (bp.dof.sqrMagnitude <= 0) continue;
            float targetRotX = bp.dof.x == 1 ? continuousActions[++i] : 0;
            float targetRotY = bp.dof.y == 1 ? continuousActions[++i] : 0;
            float targetRotZ = bp.dof.z == 1 ? continuousActions[++i] : 0;
            float jointStrength = continuousActions[++i];
            bp.SetJointTargetRotation(targetRotX, targetRotY, targetRotZ);
            bp.SetJointStrength(jointStrength);
        }
    }

    public virtual void FixedUpdate()
    {
        distanceMovedInTargetDirection += GetDistanceMovedInTargetDirection();

        var cubeForward = walkOrientationCube.transform.forward;

        // Set reward for this step according to mixture of the following elements.
        // a. Match target speed
        //This reward will approach 1 if it matches perfectly and approach zero as it deviates
        var matchSpeedReward = GetMatchingVelocityReward(cubeForward * targetWalkingSpeed, GetAvgVelocity());
        RecordStat("Reward/MatchingVelocityReward", matchSpeedReward);

        //Check for NaNs
        if (float.IsNaN(matchSpeedReward))
        {
            throw new ArgumentException(
                "NaN in moveTowardsTargetReward.\n" +
                $" cubeForward: {cubeForward}\n" +
                $" root.velocity: {bodyparts[0].rb.velocity}\n" +
                $" maximumWalkingSpeed: {maxWalkingSpeed}"
            );
        }

        // b. Rotation alignment with target direction.
        //This reward will approach 1 if it faces the target direction perfectly and approach zero as it deviates
        var headForward = head.forward;
        headForward.y = 0;
        // var lookAtTargetReward = (Vector3.Dot(cubeForward, head.forward) + 1) * .5F;
        var lookAtTargetReward = (Vector3.Dot(cubeForward, headForward) + 1) * .5F;
        RecordStat("Reward/LookAtTargetReward", lookAtTargetReward);

        //Check for NaNs
        if (float.IsNaN(lookAtTargetReward))
        {
            throw new ArgumentException(
                "NaN in lookAtTargetReward.\n" +
                $" cubeForward: {cubeForward}\n" +
                $" head.forward: {head.forward}"
            );
        }

        AddReward(matchSpeedReward * lookAtTargetReward);
    }

    //Returns the average velocity of all of the body parts
    //Using the velocity of the hips only has shown to result in more erratic movement from the limbs, so...
    //...using the average helps prevent this erratic movement
    public Vector3 GetAvgVelocity()
    {
        Vector3 velSum = Vector3.zero;

        foreach (Bodypart bp in bodyparts)
        {
            velSum += bp.rb.velocity;
        }

        var avgVel = velSum / bodyparts.Count;
        return avgVel;
    }

    //normalized value of the difference in avg speed vs goal walking speed.
    public float GetMatchingVelocityReward(Vector3 velocityGoal, Vector3 actualVelocity)
    {
        //distance between our actual velocity and goal velocity
        float upperLimit = Mathf.Max(0.1f, targetWalkingSpeed);
        var velDeltaMagnitude = Mathf.Clamp(Vector3.Distance(actualVelocity, velocityGoal), 0, upperLimit);

        //return the value on a declining sigmoid shaped curve that decays from 1 to 0
        //This reward will approach 1 if it matches perfectly and approach zero as it deviates
        float matchingVelocityReward = Mathf.Pow(1 - Mathf.Pow(velDeltaMagnitude / upperLimit, 2), 2);
        return matchingVelocityReward;
    }

    protected float GetDistanceMovedInTargetDirection()
    {
        //calculate the displacement vector
        Vector3 currentPos = root.position;
        Vector3 displacement = currentPos - previousPos;

        //project the displacement vector onto the goal direction vector
        float movementInTargetDirection = Vector3.Dot(displacement, walkOrientationCube.transform.forward);

        //update the previous position for the next frame
        previousPos = currentPos;
        return movementInTargetDirection;
    }

    protected void OnTouchedTarget()
    {
        if (lastReachedTargetTime + 0.1f <= Time.time)
        {
            lastReachedTargetTime = Time.time;
            reachedTargets++;
            onTouchedTarget.Invoke();
        }
    }

    protected void OnTouchedGround()
    {
        //check that the episode did not start in the last step to remove duplicate calls
        if (Academy.Instance.StepCount < 1) return;
        SetReward(-1f);
        EndEpisode();
    }

    protected void RecordStat(string path, float value)
    {
        if (logStats) Debug.Log($"{path}: {value}");
        statsRecorder.Add(path, value);
    }
}
\end{lstlisting}

\newpage
\section{Angepasstes Skript Zielsteuerung}
\label{code:zielsteuerung}
\begin{lstlisting}[caption={Zielsteuerung Skript},captionpos=b,label={lst:skript_zielsteuerung}]
using UnityEngine;
using Random = UnityEngine.Random;
using Unity.MLAgents;
using UnityEngine.Events;

public class TargetController1 : MonoBehaviour
{
    [Header("Target Config")]
    public float spawnRadius; //The radius in which a target can be randomly spawned.
    public bool setRandomStartPos = true;

    private Vector3 m_startingPos; //the starting position of the target

    void OnEnable()
    {
        m_startingPos = transform.position;
        if (setRandomStartPos)
        {
            MoveTargetToRandomPosition();
        }
    }

    /// <summary>
    /// Moves target to a random position within specified radius.
    /// </summary>
    public void MoveTargetToRandomPosition()
    {
        var newTargetPos = m_startingPos + (Random.insideUnitSphere * spawnRadius);
        newTargetPos.y = m_startingPos.y;
        transform.position = newTargetPos;
    }
}
\end{lstlisting}

\newpage
\section{Angepasstes Skript Orientierungsobjekt}
\label{code:orientierungsobjekt}
\begin{lstlisting}[caption={Orientationsobjekt Skript},captionpos=b,label={lst:skript_orientationsobjekt}]
using UnityEngine;

/// <summary>
/// Utility class to allow a stable observation platform.
/// </summary>
public class OrientationCubeController1 : MonoBehaviour
{
    public Transform root;
    public Transform target;

    //Update position and Rotation
    public void UpdateOrientation()
    {
        var dirVector = target.position - transform.position;
        dirVector.y = 0; //flatten dir on the y. this will only work on level, uneven surfaces
        var lookRot =
            dirVector == Vector3.zero
                ? Quaternion.identity
                : Quaternion.LookRotation(dirVector); //get our look rot to the target

        //UPDATE ORIENTATION CUBE POS & ROT
        transform.SetPositionAndRotation(root.position, lookRot);
    }

    void FixedUpdate()
    {
        UpdateOrientation();
    }
}
\end{lstlisting}

\newpage
\section{Implementierung Spherecast}
\label{code:spherecast}
\begin{lstlisting}[caption={Implementierung von Spherecast um Blickkontakt überprüfen zu können},captionpos=b,label={lst:skript_spherecast}]
void CheckLookAtTarget()
{
    RaycastHit hit;
    if (Physics.SphereCast(head.position, lookAtTargetMargin, head.forward, out hit, 9f) && hit.collider.gameObject.CompareTag("lookTarget"))
    {
        if (isLookingAtTarget)
        {
            if (startetLookingAtTarget + durationLookAtTarget <= Time.fixedTime)
            {
                onLookedAtTarget.Invoke();
                isLookingAtTarget = false;
            }
        }
        else
        {
            startetLookingAtTarget = Time.fixedTime;
            isLookingAtTarget = true;
        }
    }
    else
    {
        isLookingAtTarget = false;
    }
}
\end{lstlisting}

\newpage
\section{Implementierungen der Belohnungen für eine verbesserte Gehbewegung}
\label{code:gehbewegung_belohnung}
\begin{lstlisting}[caption={Implementierung von zusätzliche Belohnungen für Gehbewegungsanpassungen},captionpos=b,label={lst:skript_gehbewegungsbelohnungen}]
float leftFootDistance = Vector3.Distance(footL.position, target.position);
float rightFootDistance = Vector3.Distance(footR.position, target.position);
if (!leftForward && leftFootDistance < rightFootDistance)
{
    leftForward = true;
    switchTime = Time.fixedTime;
}
else if (leftForward && rightFootDistance < leftFootDistance)
{
    leftForward = false;
    switchTime = Time.fixedTime;
}
float switchDeltaTime = Time.fixedTime - switchTime;
float footSwitchReward = -Mathf.Clamp((switchDeltaTime / 4) - 0.3f, 0f, 1f);
if (logStats && footSwitchReward < 0) Debug.Log($"foot switch reward: {footSwitchReward}, switched: {switchDeltaTime} seconds ago");
RecordStat("Reward/FootSwitchReward", footSwitchReward);

float leftHandDistance = Vector3.Distance(handL.position, target.position);
float rightHandDistance = Vector3.Distance(handR.position, target.position);
float rootDistance = Vector3.Distance(root.position, target.position);
float pendulumSum = 0f;
if (leftForward)
{
    pendulumSum = (rootDistance - rightHandDistance) + (leftHandDistance - rootDistance);
}
else
{
    pendulumSum = (rootDistance - leftHandDistance) + (rightHandDistance - rootDistance);
}
float armPendulumReward = Mathf.Clamp((pendulumSum / 2) - 0.5f, -1f, 0f);
RecordStat("Reward/ArmPendulumReward", armPendulumReward);

float totalPower = GetTotalPower();
float powerSaveReward = -Mathf.Clamp(totalPower / 3000 - 0.05f, 0f, 1f);
if (logStats) Debug.Log($"power save reward: {powerSaveReward}, total: {totalPower}");
RecordStat("Reward/PowerSaveReward", powerSaveReward);

AddReward(Mathf.Min(footSwitchReward, powerSaveReward, armPendulumReward));
\end{lstlisting}

\label{code:imitation_belohnung}
\section{Implementierung von Deep Mimic Körperhaltungsbelohnung}
\begin{lstlisting}[caption={Implementierung von Deep Mimic Körperhaltungsbelohnung},captionpos=b,label={lst:skript_gehbewegungsbelohnungen}]
private float CalculatePoseReward()
{
    float sum = 0f;
    //sum over all bodyparts
    for (int i = 0; i < bodyparts.Count; i++)
    {
        Bodypart bp = bodyparts[i];
        ReferenceBodypart rbp = referenceController.referenceBodyparts[i];
        float angle = 0f;
        if (i == 0)
        {
            Vector3 bodypartUp = bp.transform.up;
            Vector3 referenceUp = rbp.transform.up;
            bodypartUp.y = 0;
            referenceUp.y = 0;
            bodypartUp.Normalize();
            referenceUp.Normalize();
            angle = Vector3.Angle(referenceUp, bodypartUp);
        }
        else
        {
            angle = Quaternion.Angle(rbp.transform.localRotation, bp.rb.transform.localRotation);
        }
        sum += angle;
    }
    float avg = sum / (bodyparts.Count - 1);
    float poseReward = -(avg / 180f);
    return poseReward;
}
\end{lstlisting}

\section{Initialisierung der Körperhaltung von Animation}
\label{code:animation_init}
\begin{lstlisting}[caption={Implementierung Initialisierung des Charakters mit Körperhaltung aus Animation},captionpos=b,label={lst:skript_körperhaltung_initialisierung}]
 public void ResetReference()
{
    transform.position = startingPos;
    //call the Play method of the Animator to start playing the animation at a specific point
    float randomPhase = Random.Range(phaseStartMin, phaseStartMax);
    animator.Play(animationName, -1, randomPhase);
    //reset reference bodyparts on next frame when animation has started from random phase
    StartCoroutine(ResetBodypartsOnNextFrame());
}

IEnumerator ResetBodypartsOnNextFrame()
{
    //wait for the next frame
    yield return null;

    // Code to be executed on next frame
    int i = 0;
    foreach (Bodypart bp in bodyparts)
    {
        Transform referenceBone = referenceController.referenceBodyparts[i].transform;
        bp.ResetTransform(referenceBone.position, referenceBone.rotation);
        i++;
    }
}
\end{lstlisting}
