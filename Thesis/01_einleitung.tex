\chapter{Einleitung}
\label{sec:einleitung}
Seit einigen Jahren steigt die Präsenz von AI und somit maschinellem Lernen in unserem Alltag. Systeme wie ChatGPT sind nicht mehr weg zu denken. Ein interessantes Feld ist dabei auch die Steuerung von Robotern, speziell Humanoiden Robotern. Die Roboter werden vorab meist in einer virtuellen Trainingsumgebung trainiert, um negative Auswirkungen wie Maschinenschaden zu verhindern und gleichzeitig die Notwendigkeit für die teure Anschaffung mehrerer Roboter zu vermeiden. Der Schritt zur Verwendung ähnlicher Systeme in der Videospielbranche ist daher nicht weit entfernt. Nvidia und Ubisoft zeigen schon seit 2020 Prototypen zur Verwendung von maschinellem Lernen im Prozess der Charaktersteuerung bzw. Charakteranimation.\cite{2022-TOG-ASE}\cite{10.1145/3355089.3356536} Im Fall von Ubisoft wird klar gezeigt, welche Komplexität das Animationssystem ihrer top Titel aufweist. Maschinelles lernen ermöglicht die Reduktion dieser Systeme in antrainierten Modellen. Speziell die Verwendung von physikalisch simulierten Charakteren ermöglicht einen Lernprozess vergleichbar mit dem eines Menschen.

Ziel der Arbeit ist es, einen Charakter physikalisch in der Unity Spieleumgebung zu simulieren. Der Charakter soll mit maschinellen Lernverfahren darauf trainiert werden, das Gleichgewicht zu halten und sich zu einem Ziel zu bewegen. Als Basis für die Trainingsumgebung soll die im Unity ML-Agents Framework entwickelte Walker Demo zum Einsatz kommen. Die Demo soll erweitert werden, sodass der Walker über Tastatureingaben gesteuert und somit in Videospielen als Ersatz für traditionelle Animationssysteme verwendet werden kann. Um die Vielfalt von Charakteranimationen abzudecken wird analysiert, wie weitere Bewegungsabläufe in das bestehende System eingefügt werden können. Außerdem soll auch die Kompatibilität mit unterschiedlichen Charaktermodellen geprüft werden.

Zu Beginn wird im Kapitel Grundlagen das Teilgebiet Verstärkendes Lernen aus dem Fachgebiet maschinelles Lernen erklärt. Darauf aufbauend wird anschließend der Aufbau, die Komponenten und Implementierungsschnittstellen der Unity ML-Agents Library aufgezeigt. Abgeschlossen werden die Grundlagen mit einer Übersicht der für die Simulation verwendeten Physikkomponenten in Unity. Auf die Grundlagen folgt eine Analyse der Walker Demo. Die Walker Demo ist eine Trainingsumgebung, welche innerhalb des Unity ML-Agents Projekts zur Demonstration der Fähigkeiten der Library entwickelt wurde. Im weiteren Verlauf der Arbeit dient diese Demonstration als Basis für den entwickelten Charaktercontroller. In Kapitel 4 wird näher auf die Entwicklung und somit den ausgeführten Versuchen sowie die darauf folgenden Ergebnisse eingegangen. Kapitel 4 teilt sich dabei in 4 Teile. In Teil 1 wird die Nutzersteuerung thematisiert. Teil 2 ermittelt, welche Anpassungen am Trainingsablauf gemacht werden müssen, um gängige Bewegungsabläufe eines Videospiel Charakters zu ermöglichen. In Teil 3 - \hl{Unterschiedliche Charaktere} - werden die Komponenten der Demo angepasst, um die Konfiguration zu vereinfachen sowie das Steuern von unterschiedlichen Charaktermodellen zu gewährleisten. Teil 4 analysiert Anpassungen, um das erlernte Gangbild natürlicher zu gestalten. Abschließend werden in Zusammenfassung und Ausblick die in der Arbeit umgesetzten Prototypen auf ihre Anwendbarkeit in Videospielen analysiert und ein Ausblick für weitere Forschungen sowie bereits existierende Forschungen in anderen Softwareumgebungen aus diesem Bereich aufgezeigt.