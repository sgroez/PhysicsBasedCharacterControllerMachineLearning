%Dokumentklasse
\documentclass[a4paper,12pt,parskip=full]{scrreprt}
\usepackage[left= 2.5cm,right = 2cm, bottom = 4 cm]{geometry}
%\usepackage[onehalfspacing]{setspace}
% ============= Packages =============

% Dokumentinformationen
\usepackage[
	pdftitle={Physik basierter Charaktercontroller mit Unity Machine Learning},
	pdfsubject={},
	pdfauthor={Simon Grözinger},
	pdfkeywords={},	
	%Links nicht einrahmen
	hidelinks
]{hyperref}


% Standard Packages
\usepackage[utf8]{inputenc}
\usepackage[ngerman]{babel}
\usepackage[T1]{fontenc}
\usepackage{graphicx, subfig}
\graphicspath{{img/}}
\usepackage{fancyhdr}
\usepackage{lmodern}
\usepackage{color}
\usepackage{enumitem}

% ============= BibLatex =============
\usepackage[backend=bibtex,style=numeric]{biblatex}
%Literaturverzeichnis
\addbibresource{Literatur.bib}

% ============= Remove new Page =============
\usepackage{etoolbox}
\makeatletter
\patchcmd{\chapter}{\if@openright\cleardoublepage\else\clearpage\fi}{}{}{}
\makeatother

% zusätzliche Schriftzeichen der American Mathematical Society
\usepackage{amsfonts}
\usepackage{amsmath}

%nicht einrücken nach Absatz
%\setlength{\parindent}{0pt}


% ============= Kopf- und Fußzeile =============
\pagestyle{fancy}
%
\lhead{}
\chead{}
\rhead{\slshape \leftmark}
%%
\lfoot{}
\cfoot{\thepage}
\rfoot{}
%%
\renewcommand{\headrulewidth}{0.4pt}
\renewcommand{\footrulewidth}{0pt}

% ============= Package Einstellungen & Sonstiges ============= 
%Besondere Trennungen
\hyphenation{De-zi-mal-tren-nung}


% ============= Dokumentbeginn =============

\begin{document}
%Seiten ohne Kopf- und Fußzeile sowie Seitenzahl
\pagestyle{empty}

\begin{center}
\begin{tabular}{p{\textwidth}}


\begin{flushright}
\includegraphics[scale=0.1]{img/logos.jpg}
\end{flushright}


\\

\begin{center}
\LARGE{\textsc{
Entwicklung eines physikbasierten Charaktercontrollers mit Unity ML Agents\\
}}
\end{center}

\\


\begin{center}
\large{\textbf{Software-Engineering}\\}
\large{Fakultät für Informatik \\
der Hochschule Heilbronn \\}
\end{center}

\\

\begin{center}
\textbf{\Large{Bachelor-Thesis}}
\end{center}

\begin{center}
vorgelegt von
\end{center}

\begin{center}
\large{\textbf{Simon Grözinger}} \\
\small{Matrikelnummer: 205047} \\
\end{center}

\end{tabular}
\end{center}

% Beendet eine Seite und erzwingt auf den nachfolgenden Seiten die Ausgabe aller Gleitobjekte (z.B. Abbildungen), die bislang definiert, aber noch nicht ausgegeben wurden. Dieser Befehl fügt, falls nötig, eine leere Seite ein, sodaß die nächste Seite nach den Gleitobjekten eine ungerade Seitennummer hat. 
\cleardoubleoddpage

% pagestyle für gesamtes Dokument aktivieren
\pagestyle{fancy}

%Inhaltsverzeichnis
\tableofcontents

%Abbildungsverzeichnis
\listoffigures

\cleardoubleoddpage

{\chapter{Einleitung}}
\label{sec:einleitung}
Machine Learning Modelle bieten neue Möglichkeiten den Prozess der Charakter animation zu erleichtern. In der Thesis soll ein Ansatz anhand bestehender Literatur und Beispiele erforscht werden, in dem Spielcharaktere physikalisch mit Rigidbodies und Joints simuliert und mit Hilfe von Machine Learning trainiert werden, um möglichst realistische Bewegung nachahmen zu können.
{\chapter{Verstärkendes Lernen}}
\label{sec:rl}
Der Begriff 'Verstärkendes Lernen' beschreibt eine Art von Problemstellung und die dafür geeigneten Problemlösungsmethoden im Bereich des Maschinellen Lernens. Die grundlegenden Bestandteile einer Trainingsumgebung sind der Agent und die Umgebung, in der der Agent seine Aktionen ausführt. Der Ansatz ist in vielerlei Hinsicht vergleichbar mit dem Lernvorgang von Menschen. Ein Baby lernt das Krabbeln ohne direkte Anweisungen, nur durch die Wahrnehmung der Umgebung, das Verhalten der Umgebung in Relation zu seinen Bewegungen und die mit den Bewegungen einhergehenden Belohnungen. Auf dieselbe Art lernt der Agent beim Verstärkenden Lernen von jedem Zustand die Aktion auszuführen, um die Belohnung zu maximieren. Im Fall des Babys sind die Belohnungen Faktoren wie Schmerz, Hunger, Müdigkeit oder gestillte Neugier. Der Agent hingegen erhält eine numerische Belohnung.\cite{sutton2018reinforcement}

\begin{figure}[htb]
  \centering  
  \includegraphics[scale=0.5]{img/rl_cycle.png}
  \caption{Verstärkendes Lernen Ablauf \protect\cite{unity_mlagents_rl_cycle}}
  \label{fig:rl_cycle}
\end{figure}

Die Abbildung 2.1 zeigt die Verbindungen zwischen dem Agent und der Umgebung. Der Agent erhält als Input einen Zustand oder meist einen Teilzustand der Umgebung und reagiert darauf mit einer Aktion. Dieser Zyklus kann je nach Problem in unterschiedlichen Intervallen durchlaufen werden. Bei kontinuierlichen Kontrollproblemen werden Aktionen meist in regelmäßigen Intervallen abgefragt. Bei rundenbasierten Spielen kann dieser Vorgang jedoch auch nur einmal pro Runde stattfinden.
{\chapter{Ml-Agents}}
\label{sec:mlagents}
Das Unity ML-Agents Toolkit ist ein Open-Source-Projekt, in dem maschinelle Lernalgorithmen und Funktionen für die Verwendung mit der Spieleumgebung Unity implementiert und kontinuierlich weiterentwickelt werden. Die Implementierung ist in zwei Bereiche unterteilt. Für die Unity-Integration ist das Paket com.unity.ml-agents aus dem Unity Asset Store zuständig. Das eigentliche Training mit den maschinellen Lernalgorithmen findet jedoch in einer separaten Python-Umgebung statt. Für die Kommunikation zwischen den beiden Bereichen verwendet das ML-Agents Toolkit eine C\# Kommunikator-Klasse, die über gRPC-Netzwerkkommunikation mit dem Python-Prozess kommuniziert. Der Python-Prozess kommuniziert über die Python Low-Level-API, die die Kommunikation übernimmt und die Befehle an den Trainer weiterleitet.\cite{unity_mlagents_toolkit_overview}

\begin{figure}[htb]
  \centering  
  \includegraphics[scale=0.4]{img/learning_environment_basic.png}
  \caption{Unity ML-Agents Lernumgebung \protect\cite{unity_mlagents_learning_environment_basic}}
  \label{fig:learning_environment_basic}
\end{figure}

Das Unity-Paket enthält zwei Komponenten: Agenten und deren Verhalten. Die Agent-Komponente bildet die Grundlage für alle Implementierungen. Sie bietet abstrakte Funktionen für die Initialisierung, den Start einer Episode, das Erfassen des Zustands der Umgebung sowie das Ausführen von Aktionen. Durch die Implementierung dieser Funktionen können unterschiedlichste Agenten entwickelt und trainiert werden. Jeder Agent ist mit einem Verhalten verknüpft, das für jede Beobachtung des Agenten eine Aktion auswählt, die der Agent ausführt. Es gibt drei Arten, wie die Verhaltensweisen agieren können. Im Lernmodus werden die Beobachtungen des Agenten für das Training und die Auswahl einer Aktion anhand des aktuellen Modells verwendet. Der Inferenzmodus nutzt hingegen ein bereits trainiertes Modell und wertet dieses aus. Der letzte Modus eines Verhaltens ist der Heuristikmodus, bei dem festgelegte Regeln im Code entscheiden, welche Aktion ausgeführt wird, ohne die Verwendung eines trainierten Modells.\cite{unity_mlagents_toolkit_overview}

\begin{figure}[htb]
  \centering  
  \includegraphics[scale=0.3]{img/learning_environment_example.png}
  \caption{Unity ML-Agents Lernumgebung Beispiel \protect\cite{unity_mlagents_learning_environment_example}}
  \label{fig:learning_environment_example}
\end{figure}

\begin{figure}[htb]
  \centering  
  \includegraphics[scale=0.5]{img/verhalten_komponente.png}
  \caption{Unity ML-Agents Verhalten Komponente}
  \label{fig:verhalten_komponente}
\end{figure}

Details zu Komponenten
Implementierungsschnittstellen

{\let\clearpage\relax\chapter{Versuche}}
\label{sec:versuche}
Text
{\chapter{Fazit}}
\label{sec:fazit}
Text

%Literaturverzeichnis
\printbibliography[title=Literaturverzeichnis]

\end{document}
