%
%  Thesis Vorlage für die Hochschule Heilbronn
%
%  Created by Prof. Dr. Detlef Stern on 2010-08-14.
%  Updated by Valentin Weber on 2020-10-05.
%  Copyright (c) 2020 . All rights reserved.
%
\documentclass[12pt,toc=bib,toc=listof]{scrreprt}
\usepackage[ngerman]{babel} 
\usepackage[utf8]{inputenc}
\usepackage[T1]{fontenc}
\usepackage{lmodern}
\usepackage{setspace}
\usepackage{geometry}

\usepackage{hyperref}
\hypersetup{
  ,colorlinks=true
  ,linkcolor=blue
  ,citecolor=blue
  ,filecolor=blue
  ,urlcolor=blue
  }

% Fachbezogene Werte (müssen aktualisiert werden)
\newcommand{\hhnsubject}{Bachelor Thesis}
\newcommand{\hhnsubjectnum}{SPO NUMMER}
\newcommand{\hhnlecturer}{Prof. Dr. Tim Reichert}

% Vom Studierenden zu aendernde Werte
\newcommand{\reprttopic}{Physik basierter Charaktercontroller mit Unity Machine Learning}
\newcommand{\reprtstudentname}{Simon Grözinger}
\newcommand{\reprtstudentid}{205047}
\urldef{\reprtstudentmail}\url{sgroezin@stud.hs-heilbronn.de}

\usepackage{ifpdf}
\ifpdf
\usepackage[pdftex]{graphicx}
\else
\usepackage{graphicx}
\fi

\usepackage[headsepline]{scrlayer-scrpage}
\pagestyle{scrheadings}
\clearscrheadfoot
\ihead{\hhnsubject: \reprttopic}
\ohead{\pagemark}
\renewcommand*{\chapterpagestyle}{scrheadings}
\renewcommand*{\chapterheadstartvskip}{}

% ============= BibLatex =============
\usepackage[backend=bibtex,style=numeric]{biblatex}
%Literaturverzeichnis
\addbibresource{Literatur.bib}

% Extra Packages
\usepackage{float}
\usepackage[table]{xcolor}
\usepackage{listings}
\lstset{
language=[Sharp]C,
captionpos=b,
numbers=left, %Nummerierung
frame=single, % Oberhalb und unterhalb des Listings ist eine Linie
showspaces=false,
showtabs=false,
breaklines=true,
showstringspaces=false,
breakatwhitespace=true,
escapeinside={(*@}{@*)},
commentstyle=\color{greencomments},
morekeywords={partial, var, value, get, set},
keywordstyle=\color{bluekeywords},
stringstyle=\color{redstrings},
basicstyle=\ttfamily\small,
literate=%
  {Ö}{{\"O}}1
  {Ä}{{\"A}}1
  {Ü}{{\"U}}1
  {ß}{{\ss}}1
  {ü}{{\"u}}1
  {ö}{{\"o}}1
  {ä}{{\"a}}1
  }
\usepackage{tikz}
\usetikzlibrary{shapes.geometric, arrows, fit}
\pgfdeclarelayer{bg}
\pgfsetlayers{bg,main}
\tikzstyle{rounded} = [rectangle, rounded corners, minimum width=3cm, minimum height=1cm,text centered, draw=black]
  
\definecolor{bluekeywords}{rgb}{0,0,1}
\definecolor{greencomments}{rgb}{0,0.5,0}
\definecolor{redstrings}{rgb}{0.64,0.08,0.08}
\definecolor{xmlcomments}{rgb}{0.5,0.5,0.5}
\definecolor{types}{rgb}{0.17,0.57,0.68}


% Deckblatt Definitionen (begin)
\titlehead{\flushright\includegraphics[scale=0.1]{./img/hhn.png}}
\subject{{\hhnsubject{} (\hhnsubjectnum{})}}
\title{\reprttopic}
\author{\reprtstudentname\footnote{\reprtstudentid, \reprtstudentmail}}
%% Datum nie auf einen festen Wert setzen
\publishers{Eingereicht bei \hhnlecturer}
% Deckblatt Definitionen (end)

\begin{document}
\pagenumbering{Roman} 
\selectlanguage{ngerman}
\maketitle
\newgeometry{left=30mm, top=25mm, right=15mm, bottom=25mm}

\tableofcontents

\addchap{Abkürzungsverzeichnis} % (fold)
\label{sec:abkuerzungsverzeichnis}

\begin{description}
  \item[ABK:] ABKÜRZUNG 
\end{description}

% chapter abkuerzungsverzeichnis (end)

%Abbildungsverzeichnis
\listoffigures

%Tabellenverzeichnis
\listoftables

%Codesnippet-verzeichnis
\lstlistoflistings

\newpage
\pagenumbering{arabic}

{\chapter{Einleitung}}
\label{sec:einleitung}
Machine Learning Modelle bieten neue Möglichkeiten den Prozess der Charakter animation zu erleichtern. In der Thesis soll ein Ansatz anhand bestehender Literatur und Beispiele erforscht werden, in dem Spielcharaktere physikalisch mit Rigidbodies und Joints simuliert und mit Hilfe von Machine Learning trainiert werden, um möglichst realistische Bewegung nachahmen zu können.

\chapter{Grundlagen}
\label{sec:grundlagen}
Dieses Kapitel behandelt die Grundlagen der verwendeten Technologien, Paketen und Unity Komponenten.
{\section{Verstärkendes Lernen}}
\label{sec:rl}
Der Begriff 'Verstärkendes Lernen' beschreibt eine Art von Problemstellung und die dafür geeigneten Problemlösungsmethoden im Bereich des Maschinellen Lernens. Die grundlegenden Bestandteile einer Trainingsumgebung sind der Agent und die Umgebung. Die Umgebung kann sich unabhängig vom Agent verändern. Der Agent hat mit den Aktionen aber Einfluss auf die Umgebung. 

Es ist in vielerlei Hinsicht vergleichbar mit dem Lernvorgang von Menschen. Ein Baby lernt das Krabbeln ohne direkte Anweisungen. Es bewegt und agiert in der Umgebung. Es beobachtet wie die Umgebung auf das Verhalten reagiert. Der damit einhergehende eigene Gefühlszustand und externe Einflüsse werden als Rückmeldung evaluiert. Durch die Rückmeldung wird das Verhalten entweder an- oder abtrainiert. Auf dieselbe Art lernt der Agent beim Verstärkenden Lernen von jedem Zustand die Aktion auszuführen, um die Belohnung zu maximieren. Die Belohnung können dabei positiv oder negativ sein. Im Fall des Babys sind die Belohnungen Faktoren wie Schmerz, Hunger, Müdigkeit, gestillte Neugier oder Lob von Mitmenschen. Der Agent hingegen erhält eine numerische Belohnung.\cite{sutton2018reinforcement}

\begin{figure}[H]
  \centering
  \begin{tikzpicture}[node distance=3cm]
    \node(agent) [rounded, draw=blue, fill=blue!80, text=white]{Agent};
    \node(umgebung) [rounded, draw=blue, fill=blue!80, text=white, below of=agent]{Umgebung};
    
    \draw [-latex, line width=0.3mm] ([xshift=-1cm] umgebung.north) -- ([xshift=-1cm] agent.south) node[midway, right] {Belohnung};
    
    \path (0,0) (umgebung) -- (4,1) (agent) coordinate[pos=0.5](aktion);
    \draw [-latex, line width=0.3mm] (umgebung) -| (aktion) node[pos=0.8, right] {Aktion} |- (agent);
    
    \path (0,0) (umgebung) -- (-4,1) (agent) coordinate[pos=0.5](aktion);
    \draw [-latex, line width=0.3mm] (umgebung) -| (aktion) node[pos=0.8, left] {Beobachtung} |- (agent);
  \end{tikzpicture}
  \caption{Verstärkendes Lernen Ablauf}
  \label{fig:vl_ablauf}
\end{figure}

Die Abbildung \ref{fig:vl_ablauf} zeigt die Verbindungen zwischen dem Agent und der Umgebung. Der Agent erhält als Input einen Zustand oder meist einen Teilzustand der Umgebung und reagiert darauf mit einer Aktion. Dieser Zyklus kann je nach Problem in unterschiedlichen Intervallen durchlaufen werden. Bei kontinuierlichen Kontrollproblemen werden Aktionen meist in regelmäßigen Intervallen angefragt. Bei Problemen mit einem festgelegten Ablauf kann dieser Vorgang jedoch auch nur in einer bestimmten Phase stattfinden.

\newpage
{\section{Ml-Agents}}
\label{sec:mlagents}
Das Unity ML-Agents Toolkit ist ein Open-Source-Projekt, in dem maschinelle Lernalgorithmen und Funktionen für die Verwendung mit der Spieleumgebung Unity implementiert und kontinuierlich weiterentwickelt werden. 

\subsection{Aufbau}
Die Implementierung ist in zwei Bereiche unterteilt. Für die Unity-Integration ist das Paket com.unity.ml-agents aus dem Unity Asset Store zuständig. Das eigentliche Training mit den maschinellen Lernalgorithmen findet jedoch in einer separaten Python-Umgebung statt. Für die Kommunikation zwischen den beiden Bereichen verwendet das ML-Agents Toolkit eine gRPC-Netzwerkkommunikation, worüber Zustand der Simulationsumgebung in Unity, ausgewählte Aktionen des neuronalen Netzes in Python und weitere Werte für die Auswertung des Trainings ausgetauscht werden.\cite{juliani2020}

\begin{figure}[H]
  \centering  
  \begin{tikzpicture}[node distance=2cm]
    \node(umgebung) [draw] {Umgebung};
    \node (agent1) [rounded, draw=green, fill=green!30, below of=umgebung, xshift=0.75cm, yshift=0.8cm] {Agent 1};
    \node (agent2) [rounded, right of=agent1, xshift=2cm, draw=green, fill=green!30] {Agent 2};
    \node (agent3) [rounded, right of=agent2, xshift=2cm, draw=green, fill=green!30] {Agent 3};
    \node (akademie) [rounded, below of=agent2 , draw=yellow, fill=yellow!30] {Akademie};

    \begin{pgfonlayer}{bg}
      \node(umgebung_bg) [draw, fill=black!20, inner sep=10pt, fit=(agent1) (agent2) (agent3) (akademie)] {};
    \end{pgfonlayer}

    \node (python_api) [rounded, below of=akademie, draw=orange, fill=orange!30] {Python API};
    \node (python_trainer) [rounded, right of=python_api, xshift=2cm, draw=red, fill=red!30] {Python Trainer};


    \draw [latex-latex, line width=0.3mm] (agent1)  |- node {} (akademie);
    \draw [latex-latex, line width=0.3mm] (agent2) -- (akademie);
    \draw [latex-latex, line width=0.3mm] (agent3) |- node {} (akademie);

    \draw [latex-latex, line width=0.3mm] (akademie) -- (python_api);
    \draw [latex-latex, line width=0.3mm] (python_api) -- (python_trainer);
  \end{tikzpicture}
  \caption{Unity ML-Agents Aufbau}
  \label{fig:mlagents_aufbau}
\end{figure}


Das Unity-Paket enthält drei Grundlegenden Komponenten, die Akademie, Agenten und Sensoren.
Das Unity-Paket enthält zwei Komponenten: Agenten und deren Verhalten. Die Agent-Komponente bildet die Grundlage für alle Implementierungen. Sie bietet abstrakte Funktionen für die Initialisierung, den Start einer Episode, das Erfassen des Zustands der Umgebung sowie das Ausführen von Aktionen. Durch die Implementierung dieser Funktionen können unterschiedlichste Agenten entwickelt und trainiert werden. Jeder Agent ist mit einem Verhalten verknüpft, das für jede Beobachtung des Agenten eine Aktion auswählt, die der Agent ausführt. Es gibt drei Arten, wie die Verhaltensweisen agieren können. Im Lernmodus werden die Beobachtungen des Agenten für das Training und die Auswahl einer Aktion anhand des aktuellen Modells verwendet. Der Inferenzmodus nutzt hingegen ein bereits trainiertes Modell und wertet dieses aus. Der letzte Modus eines Verhaltens ist der Heuristikmodus, bei dem festgelegte Regeln im Code entscheiden, welche Aktion ausgeführt wird, ohne die Verwendung eines trainierten Modells.\cite{juliani2020}

\subsection{Komponenten}
In diesem Kapitel werde Ich die Grundlegenden Komponenten des Unity ML-Agents Packets, welche in der Arbeit verwendet wurden erklären. Darurch sollten Codeausschnitte und Komponentenabbildungen in folgenden Kapiteln deutlich zu verstehen sein.

\subsubsection{Verhalten}
\begin{figure}[H]
  \centering  
  \includegraphics[scale=0.5]{img/verhalten_komponente.png}
  \caption{Unity ML-Agents Verhalten Komponente}
  \label{fig:verhalten_komponente}
\end{figure}

\begin{table}[H]
\centering
{\rowcolors{2}{lightgray}{gray!50!lightgray!50}
\begin{tabular}{ |p{4cm}|p{8cm}| }
\hline
Konfigurationsfeld& Beschreibung \\
\hline
Behaviour Name & Name des Verhaltens / wird in Trainer Konfiguration referenziert \\
Space Size & Anzahl an Beobachtungen / Inputknoten für NN \\
Continuous Actions & Anzahl an Aktionen / Outputknoten von NN \\
Model & Referenz auf bereits trainiertes Modell zur Verwendung in Inferenz \\
Behaviour Type & Lernmodus Default = Lernen, Heuristic, Inferenz \\
\hline
\end{tabular}}
\caption{Test}
\end{table}

\subsubsection{Entscheidung}
\begin{figure}[H]
  \centering  
  \includegraphics[scale=0.5]{img/entscheidung_anfragen_komponente.png}
  \caption{Unity ML-Agents Entscheidung Anfragen Komponente}
  \label{fig:entscheidung_anfragen_komponente}
\end{figure}

\begin{center}
{\rowcolors{2}{lightgray}{gray!50!lightgray!50}
\begin{tabular}{ |p{4cm}|p{8cm}| }
\hline
Konfigurationsfeld& Beschreibung \\
\hline
Decision Period & Anzahl an Akademie-Schritten (standard ein Schritt pro Physikupdate) bis zur nächsten Entscheidung \\
Take Actions Between Decisions &  Kontrollkasten ob Agent Aktionen zwischen Entscheidungen ausführen soll \\
\hline
\end{tabular}}
\end{center}


\subsection{Programmierschnittstellen}
\begin{lstlisting}[caption={Academy Instanzvariablen},captionpos=b]
envParams = Academy.Instance.EnvironmentParameters;
statsRecorder = Academy.Instance.StatsRecorder;
\end{lstlisting}
Die Akademie stellt mit dem Attribut EnvironmentParameters die Umgebungsparameter aus Trainer Konfiguration oder aktueller Lektion bereit

Mit dem StatsRecorder lassen sich Daten aggregieren um diese nach oder während dem Training über die Tensorboard Visualisierung auszuwerten

\begin{lstlisting}[caption={Agent Funktionen},captionpos=b]
public override void CollectObservations(VectorSensor sensor)
{
    sensor.AddObservation(floatObservation);
}

public override void OnActionReceived(ActionBuffers actionBuffers)
{
    var continuousActions = actionBuffers.ContinuousActions;
    float action = continuousActions[0]
}

public virtual void FixedUpdate()
{
    AddReward(floatReward);
}
\end{lstlisting}

In der CollectObservations Methoden wird festgelegt welche Daten dem Agent für das Training bereit stehen, dieser Schritt wird für jede angefragte Entscheidung ausgeführt und das Ergebnis an das NN Modell oder den Python Trainer übergeben.

Wenn eine Entscheidung angefragt wurde und das NN Modell ein Ergebnis liefert wird dieses hier von numerischen Werten in Aktionen umgewandelt.

Im folgenden Beispielcode wird ein Reward in jedem FixedUpdate vergeben über die AddReward Methode die auch Teil der Agenten-Komponente ist. Der Reward kann aber an jeder Stelle im Code vergeben werden, der Code dient hier nur als ein Beispiel.

Die Trainings Konfigurationdatei enthält mehrere Teile. Der hyperparameter Teil enthält die Hyperparameter des Maschinellen Lernalgorithmuses, danach folgt der network\_settings Teil welcher die Konfiguration des Neuronalennetzes festlegt. Anschließend folgen noch Konfigurationen für die Belohnungssignale im Bereich reward\_signals und Einstellungen für die Speicherung der Daten sowie der länge des Trainings. Ganz am Ende der Konfigurationsdatei befinden sich noch Umgebungsparameter welche erweitert und während dem Training ausgelesen werden können.
\begin{lstlisting}[caption={Trainer Konfigurationsdatei},captionpos=b]
{
behaviors:
  Walker:
    trainer_type: ppo
    hyperparameters:
      batch_size: 2048
      buffer_size: 20480
      learning_rate: 0.0003
      beta: 0.005
      epsilon: 0.2
      lambd: 0.95
      num_epoch: 3
      learning_rate_schedule: linear
    network_settings:
      normalize: true
      hidden_units: 256
      num_layers: 3
      vis_encode_type: simple
    reward_signals:
      extrinsic:
        gamma: 0.995
        strength: 1.0
    keep_checkpoints: 5
    checkpoint_interval: 5000000
    max_steps: 30000000
    time_horizon: 1000
    summary_freq: 30000
environment_parameters:
  environment_count: 100.0
}
\end{lstlisting}

{\section{Unity Physik}}
\label{sec:physik}

\chapter{Analyse}
\label{sec:analyse}
Zusätzlich zu den maschinellen Lernkomponenten stellt Unity auch Demonstrationsumgebungen bereit, in denen verschiedene Lösungen für gängige Verstärkungslernprobleme implementiert sind. In der Walker-Demo wird ein physisch simulierter Charakter darauf trainiert, zu einem Zielwürfel zu laufen. Sie implementiert bereits einige grundlegende Steuerungsmechanismen, die erforderlich sind, um einen Charakter in einer Umgebung zu bewegen und zu kontrollieren. Aus diesem Grund wird in dieser Arbeit die Walker-Demo als Basis für die Entwicklung genutzt. 

In diesem Kapitel wird daher die Walker-Demo analysiert, um in den folgenden Kapiteln darauf aufzubauen. Es wird untersucht wie die Lernumgebung aufgebaut ist. Anschließend wird der Ablauf und die Komponenten für das verstärkende Lernen analysiert. Zum Abschluss wird das Trainingsergebnis und die Bewegungsabläufe der Demo analysiert.
\section{Lernumgebung}
Die Umgebung besteht aus einem quadratischen Spielfeld mit einem Boden und Wänden, die der Charakter nicht verlassen kann (siehe Abbildung \ref{fig:walker_aufbau}). Diese Begrenzungen dienen dazu, die Bewegung des Charakters zu kontrollieren und sicherzustellen, dass die Lernumgebung konsistent bleibt. Die Umgebung umfasst weiterhin den Läufer und das Ziel.

\begin{figure}[H]
  \centering  
  \includegraphics[scale=0.35]{img/walker_aufbau.png}
  \caption{Walker-Demo Umgebung}
  \label{fig:walker_aufbau}
\end{figure}

Der Läufer besteht aus einfachen geometrischen Körpern. Insgesamt 11 Kapseln, drei Kugeln und zwei Quadern von welchen jede über eine Festkörper- und eine Kollisions-Physikkomponente verfügt. Die Gelenke zwischen den Körperteilen werden als Kugelgelenke simuliert, um eine flexible und natürliche Bewegung zu gewährleisten. Die genaue Physikkonfiguration der Körperteile wird in der Tabelle \ref{table:walker_körperteile} veranschaulicht. Diese Konfiguration spielt eine zentrale Rolle, da die gesetzten Freiheiten sowie Einschränkungen beeinflussen wie der Läufer lernt, auf das Ziel zuzulaufen.

\begin{figure}[H]
  \centering  
  \includegraphics[scale=0.35]{img/läufer.png}
  \caption{Walker-Demo Läufer}
  \label{fig:läufer}
\end{figure}

\begin{table}[H]
  \centering
  {\rowcolors{1}{gray!10}{white}
  \begin{tabular}{ |p{3cm}|p{3cm}|p{2cm}|p{4cm}|p{2cm}| }
  \hline
  \textbf{Körpertei}l& \textbf{Verbundenes Körperteil} & \textbf{Gewicht} & \textbf{Winkellimits} & \textbf{Form} \\
  \hline
  Hüfte & - & 15kg & - & Kapsel \\
  \hline
  Wirbelsäule & Hüfte & 10kg & x(-20,20) y(-20,20) z(-15,15) & Kapsel \\
  \hline
  Oberkörper & Wirbelsäule & 8kg & x(-20,20) y(-20,20) z(-15,15) & Kapsel \\
  \hline
  Kopf & Oberkörper & 6kg & x(-30,10) y(-20,20) & Kugel \\
  \hline
  Oberarm LR & Oberkörper & je 4kg & x(-60,120) y(-100,100) & Kapsel \\
  \hline
  Unterarm LR & Oberarm & je 3kg & x(0,160) & Kapsel \\
  \hline
  Hand LR & Unterarm & je 2kg & - & Kugel \\
  \hline
  Oberschenkel LR & Hüfte & je 14kg& x(-90,60) y(-40,40) & Kapsel \\
  \hline
  Unterschenkel LR & Oberschenkel & je 7kg &  x(0,120) & Kapsel \\
  \hline
  Fuß LR & Unterschenkel & je 5kg & x(-20,20) y(-20,20) z(-20,20) & Quader \\
  \hline
  \end{tabular}}
  \caption{Walker Agent Körperteile}
  \label{table:walker_körperteile}
\end{table}

Das Walker Agent Skript, definiert den Läufer als Agent für das maschinelle Lernen. In Abbildung \ref{fig:agent_konfiguration} wird die Agentenkomponente im Inspektor gezeigt. Diese Komponente ist entscheidend für die Konfiguration des Läufers. Um die Komponente zu nutzen, müssen hier die Körperteile des Walkers referenziert werden. Das Walker Agent Skript registriert die Körperteile bei der Initialisierung in der Gelenk Motor Steuerung, wodurch eine effektive Schnittstelle zur Kontrolle der Gelenke geschaffen wird. Die Gelenk Motor Einstellungen (Joint Drive Settings) siehe Abbildung \ref{fig:agent_konfiguration} \hl{bestimmen die Stärke mit welcher die Gelenke in die Zielstellung bewegt werden.} Zusätzlich kann eine Zielgeschwindigkeit festgelegt werde und ob die Geschwindigkeit variieren soll während dem Training. Die Geschwindigkeit während dem Training zu variieren hilft dem Agent sein Verhalten besser an Umgebungsveränderungen anzupassen. Als letztes muss auch das Zielobjekt referenziert werden.

\begin{figure}[H]
  \centering  
  \includegraphics[scale=0.5]{img/gelenk_motor_steuerung.png}
  \caption{Gelenk Motor Steuerung}
  \label{fig:gelenk_motor_steuerung}
\end{figure}

\begin{itemize}
  \item Max Joint Spring: Bestimmt den Drehmoment mit welchem das Gelenk in die Zielposition rotiert wird.
  \item Joint Dampen: Verringert den Drehmoment proportional zur Differenz zwischen aktueller Geschwindigkeit und der Zielgeschwindigkeit. Verringert Schwingungen.
  \item Max Joint Force Limit: Gibt die maximale Kraft des Gelenks an (verhindert zu schnelle Bewegung bei großer Abweichung).
\end{itemize}

\begin{figure}[H]
  \centering  
  \includegraphics[scale=0.5]{img/agent_konfiguration.png}
  \caption{Agent Konfiguration}
  \label{fig:agent_konfiguration}
\end{figure}

\section{Training}

Zu beginn werden die Körperteile in der Gelenk Motor Steuerung initialisiert und das Ziel auf eine zufällige Position gesetzt.
Darauf folgend beginnt die Simulation der Trainingsepisoden. Hierfür werden alle Körperteile in ihre Startposition gesetzt und die Rotation des Läufers um die Y Achse zufällig gesetzt. Die zufällige Rotation hilf dabei das Verhalten des Läufers flexibel zu gestalten. Es wird weiterhin eine zufällige Zielgeschwindigkeit gewählt um zusätzliche Stabilität zu gewährleisten.

Sind die Vorbereitungen getroffen beginnt die Simulation mit einer Updatefrequenz von 75 Hz für die Phsikkalkulation. Der Agent fragt jedes fünfte Physikupdate eine Entscheidung an. Bei der Simulationsfrequenz von 75 Hz ergibt das eine Frequenz von 15 Anfragen pro Sekunde. Der Grund warum die Anfragen nur jedes fünfte Update angefragt werden ist das der Agent durch diese Einschränkung seine Bewegungen genauer wählen muss. Es kann so verhindert werden das der Agent zu hastige und ruckartige Bewegungswechsel lernt. Sobald eine Entscheidung angefragt ist, erfasst der Agent den Zustand der Umgebung. Dieser wird anschließend im referenzierten Verhalten ausgewertet und eine Aktion ausgewählt.

Die Beobachtung des Agenten wird in Tabelle \ref{table:walker_beobachtung} dargestellt. Für jedes Körperteil wird die Beobachtung aus Tabelle \ref{table:walker_beobachtung_körperteil} dem Zustand angefügt. Die Beobachtungen müssen den Zustand des Läufers und der Umgebung im Bezug auf das Trainingsziel genau darstellen. Nur so kann der Agent die Situation verstehen und eine passende Aktion auswählen und gleichermaßen sein Verhalten optimieren.

\begin{table}[H]
  \centering
  {\rowcolors{1}{gray!10}{white}
  \begin{tabular}{ |p{1cm}|p{9cm}|p{5cm}|}
  \hline
  \textbf{ID} & \textbf{Beobachtung} & \textbf{Anmerkung}  \\
  \hline
  \rowids & Abweichung Durchschnittsgeschwindigkeit von Zielgeschwindigkeit &  \\
  \hline
  \rowids & Durchschnittsgeschwindigkeit &  \\
  \hline
  \rowids & Zielgeschwindigkeit & \\
  \hline
  \rowids & Abweichung Hüftrotation von Zielrotation & \\
  \hline
  \rowids & Abweichung Kopfrotation von Zielrotation & \\
  \hline
  \rowids & Zielposition & \\
  \hline
  \rowids & Körperteil Beobachtungen & Beobachtung aus Tabelle \ref{table:walker_beobachtung_körperteil} für jedes Körperteil \\
  \hline
  \end{tabular}}
  \caption{Walker Agent Beobachtung}
  \label{table:walker_beobachtung}
\end{table}
\rowidsclear

\begin{table}[H]
  \centering
  {\rowcolors{1}{gray!10}{white}
  \begin{tabular}{ |p{1cm}|p{9cm}|p{5cm}|}
  \hline
  \textbf{ID} & \textbf{Beobachtung} & \textbf{Anmerkung}  \\
  \hline
  \rowids & Bodenkontakt & \\
  \hline
  \rowids & Geschwindigkeit & \\
  \hline
  \rowids & Rotationsgeschwindigkeit & \\
  \hline
  \rowids & Position relativ zur Hüfte & \\
  \hline
  \rowids & LokaleRotation & Fehlt für Hüfte und Hände \\
  \hline
  \rowids & Gelenkstärke & Fehlt für Hüfte und Hände \\
  \hline
  \end{tabular}}
  \caption{Walker Agent Körperteil Beobachtung}
  \label{table:walker_beobachtung_körperteil}
\end{table}
\rowidsclear

Das Format einer Aktion besteht aus den in Tabelle \ref{table:walker_aktion} aufgeführten Feldern für jedes Körperteil des Läufers, ausgenommen der Hüfte und Hände. Jedes Körperteil wird somit separat bewegt, um die Bewegungen zu optimieren und schlussendlich das Gleichgewicht zu halten und das Fortbewegen zu erlernen.

Die Hüfte ist das zentrale Körperteil woran alle weiteren Körperteile mit Gelenken direkt oder indirekt anknüpfen. Aufgrund dieser zentralen Rolle wird die Hüftbeugung über das Gelenk des verbundenen Körpers gesteuert.

Da die Hände kaum Relevanz für das laufen haben, sind in der Demo fest mit dem Unterarm verbunden und brauchen daher nicht gesteuert werden.

\begin{table}[H]
  \centering
  {\rowcolors{1}{gray!10}{white}
  \begin{tabular}{ |p{1cm}|p{9cm}|p{5cm}|}
  \hline
  \textbf{ID} & \textbf{Beobachtung} & \textbf{Anmerkung}  \\
  \hline
  \rowids & Rotationswinkel X & Nur wenn Körperteil X Rotation beweglich ist\\
  \hline
  \rowids & Rotationswinkel Y & Nur wenn Körperteil Y Rotation beweglich ist\\
  \hline
  \rowids & Rotationswinkel Z & Nur wenn Körperteil Z Rotation beweglich ist\\
  \hline
  \rowids & Gelenkstärke & \\
  \hline
  \end{tabular}}
  \caption{Walker Agent Aktion}
  \label{table:walker_aktion}
\end{table}
\rowidsclear

Nach dem Erhalten der Aktion werden über die Gelenk Motor Steuerung die Zielrotationen sowie die Maximale Kraft des Gelenks festgelegt, und somit der Läufer gesteuert.

Die Belohnungsfunktion enthält zwei Komponenten. Zum einen wird die Differenz der Bewegung in Zielrichtung zwischen momentaner Bewegung und Zielbewegung durch die Funktion $R_V$ bewertet. Somit wird der Läufer dazu motiviert effizient auf das Ziel zuzusteuern, indem Geschwindigkeit und Richtung optimiert werden. Zum Anderen wird die Abweichung zwischen momentaner Blickrichtung und der Zielrichtung in $R_L$ berechnet. Diese Komponente stellt sicher das der Läufer sich vorwärts geradeaus auf das Ziel bewegt. Die Belohnung ergibt sich am ende durch die Multiplikation beider Teilterme. Die Verwendung der Multiplikation hat zur Folge das die Belohnung gleichermaßen von beiden Teiltermen abhängig ist und es somit notwendig ist, beide Teile gleichzeitig zu optimieren. Als Ergebnis lernt der Läufer gleichermaßen die Ausrichtung als auch die Bewegung in Zielrichtung.

\begin{figure}[H]
  \centering  
  \includegraphics[scale=0.5]{img/match_velocity_demo_vel1.png}
  \caption{Walker Demo Match Velocity Belohnungsfunktion}
  \label{fig:match_velocity_demo_vel1}
\end{figure}

$V_\delta=Clip(|\vec{Geschwindigkeit} - \vec{Zielgeschwindigkeit}|, 0, |\vec{Zielgeschwindigkeit}|)$ \\
$R_V=(1 - (V_\delta / |\vec{Zielgeschwindigkeit}|)^2)^2$ \\

\begin{figure}[H]
  \centering  
  \includegraphics[scale=0.5]{img/look_at_target_demo.png}
  \caption{Walker Demo Look At Target Belohnungsfunktion}
  \label{fig:look_at_target_demo}
\end{figure}

$R_L=(\vec{Zielrichtung} \cdot \vec{Blickrichtung})+ 1) \cdot 0.5$ \\
$R=R_V \cdot R_L$

Die Belohnung wird in jedem Physikupdate neu berechnet und dem Agenten hinzugefügt. Die Belohnung wird für den Zeitraum zwischen zwei Entscheidungen aufsummiert. Bevor die nächste Entscheidung getroffen wird, wird das Tupel aus Beobachtung, Aktion und erhaltene Belohnung im Trainingsbuffer gespeichert. Hat der Buffer genug Informationen gespeichert beginnt ein Lernprozess in welchem die zuvor gespeicherten Tupel evaluiert werden. Es wird der PPO Algorithmus auf Teilbatches ausgeführt und so schrittweise das Verhalten angepasst.

Erreicht der Läufer ein Ziel wird dieses an eine neue zufällige Position in der Umgebung bewegt.

Die Trainingsepisode läuft solange bis entweder 5000 Schritte erreicht sind oder der Läufer fällt. Wenn ein Körperteil des Läufers, ausgenommen den Füßen und Schienbeinen, den Boden berührt wird die Trainingsepisode sofort beendet. Diese Technik nennt man frühes stoppen, es wird verwendet um zu verhindern das der Agent einen Großteil des Trainings in zuständen verbringt welche für das eigentliche Ziel keinen Mehrwert bieten. Fällt der Läufer benötigt es eine sehr komplexe Reihenfolge an Aktionen zum zurück auf die Beine zu kommen. Im Ende bringt es den Läufer aber kein bisschen weiter an das eigentliche Ziel. Ist die Episode zu Ende wird sofort eine neue gestartet.

\section{Auswertung}
Der Agent der Walker Demo erlernt über 30 Millionen Trainingsschritte ein Verhalten welches beinahe die Grenze der Episodenlänge erreicht ohne zu fallen. Dabei erreicht es zudem eine durchschnittliche Belohnung pro Episode von 1600 (siehe Abbildung \ref{fig:analyse_training}). Die durchschnittliche Belohnung ist ohne Kontext erstmal nur eine Zahl. Schaut man jedoch genauer ergibt sich die durchschnittliche Belohnung aus der durchschnittlichen Episodenlänge angegeben mit der Anzahl an angefragten Entscheidungen. Mit 800 Anfragen pro Episode kommt man auf 4000 Trainingsschritte beziehungsweise Physikupdates. Teilt man nur die durchschnittliche Belohnung mit der Anzahl an Schritten kommt man auf eine durchschnittliche Belohnung von ca. 0.4 pro Schritt. Die im Verlauf der Arbeit hinzugefügten Statistiken der Belohnungen zeigen eine Aufteilung von 0.9 Belohnung für die Blickrichtung und 0.45 für das halten der Zielgeschwindigkeit (siehe Abbildung \ref{fig:analyse_training_belohnung}. Eine Belohnung von 0.9 der Blickrichtungsbelohnung ergibt eine Abweichung von durchschnittliche 35 Grad. Die Zielgeschwindigkeitsbelohnung ergibt eine Abweichung von ca. 51\%.

\begin{figure}[H]
  \centering  
  \includegraphics[scale=0.5]{img/analyse_training.png}
  \caption{Walker Demo Analyse Trainingsgraphen}
  \label{fig:analyse_training}
\end{figure}

\begin{figure}[H]
  \centering  
  \includegraphics[scale=0.5]{img/analyse_training_belohnung.png}
  \caption{Walker Demo Analyse Training Belohnungsgraphen}
  \label{fig:analyse_training_belohnung}
\end{figure}

\begin{figure}[H]
  \centering  
  \includegraphics[scale=0.5]{img/analyse_training2.png}
  \caption{Walker Demo Analyse Training Zielgraphen}
  \label{fig:analyse_training_2}
\end{figure}

Der Läufer lernt die Belohnungen nicht perfekt sondern optimiert die länge der Episode, was auch zu steigender Belohnung innerhalb der Episode führt. In Abbildung \ref{fig:analyse_training_2} zu sehen läuft er während einer Episode durchschnittlich eine Distanz von 80 Einheiten und erreicht dabei 10.4 Ziele. Der Läufer lernt sehr stabil sich zum Ziel zu bewegen. In Abbildung \ref{fig:analyse_gangbild} ist das Gangbild des Läufers abgebildet. Das Gangbild wechselt periodisch das Standbein, setzt den einen Fuß vor den anderen und drückt sich über das Standbein voran. Die Arme jedoch schwingt der Läufer sehr stark um das Gleichgewicht halten zu können. Das Gangbild ist daher nicht ganz wie von einem zweibeinigen Menschenähnlichen Charakter erwartet.

\begin{figure}[H]
  \centering
  \begin{tabular}{ccc}
    \includegraphics[width=0.27\textwidth]{img/walker_laufen1.png} & \includegraphics[width=0.27\textwidth]{img/walker_laufen2.png} & \includegraphics[width=0.27\textwidth]{img/walker_laufen3.png}\\
    \includegraphics[width=0.27\textwidth]{img/walker_laufen4.png} & \includegraphics[width=0.27\textwidth]{img/walker_laufen5.png} & \includegraphics[width=0.27\textwidth]{img/walker_laufen6.png}\\
  \end{tabular}
  \caption{Walker Demo Analyse Gangbild}
  \label{fig:analyse_gangbild}
\end{figure}
  

\chapter{Charaktercontroller}
\label{sec:charaktercontroller}
Folgender Abschnitt geht auf die Anforderungen eines Charaktercontrollers so wie die Entwicklung innerhalb dieser Arbeit ein. Dabei werden verschiedenen Ansätze getestet, implementiert und evaluiert. Die Versuche sind jeweils in Ziel, Umsetzung und Auswertung unterteilt.

\section{Nutzersteuerung}
Um von einem Charaktercontroller sprechen zu können, muss der Agent über Nutzerinput gesteuert werden. Mit dieser Anforderung wurden die ersten Anpassungen der Walker Demo implementiert.

\subsection{Versuch 1}
\label{subsec:versuch_1}
Das Ziel soll zur Laufzeit durch Tastatureingabe des Nutzers gesteuert werden können.
Um das zu umzusetzen wird die Tastatureingabe eingelesen und darauf basierend das Ziel relativ zur Hüfte des Walkers gesetzt. Der Input bestimmt dabei die Distanz zum Läufer auf den Richtungsvektoren des lokalen Koordinatensystems der Hüfte (siehe Listing \ref{lst:nutzersteuerung_1}). Mit den Achsen Vorwärts und Rechts zur Hüfte und dem Input horizontal wie vertikal von -1 bis 1 kann der Läufer in alle vier Richtungen und diagonal bewegt werden.
\begin{lstlisting}[caption={Nutzersteuerung erster Prototyp},captionpos=b,label={lst:nutzersteuerung_1}]
void FixedUpdate()
{
    //Einlesen Tastatur Input
    float inputHor = Input.GetAxis("Horizontal");
    float inputVert = Input.GetAxis("Vertical");
        
    //Setzen der Zielposition
    transform.position = root.position + root.forward * inputVert + root.right * inputHor;
}
\end{lstlisting}
Diese Implementierung ermöglicht, dass der Läufer über die Tastatureingabe gesteuert werden kann, hat aber noch einige Probleme. Das positionieren des Ziels relativ zur Hüfte hat als Konsequenz das jede Bewegung der Hüfte Einfluss auf die Laufrichtung hat, wodurch sich der Läufer nicht stabil steuern lässt. 

\subsection{Versuch 2}
Um das Problem aus \ref{subsec:versuch_1} zu beheben wird in folgendem Versuch das Ziel relativ zu den Weltachsen gesetzt. In Unity gibt die Vector3 Klasse mit den Feldern forward den Vektor (0,0,1) und mit right den Vektor (1,0,0) in Weltkoordinaten an.
\begin{lstlisting}[caption={Nutzersteuerung berechnung mit Weltachsen},captionpos=b,label={lst:nutzersteuerung_2}]
//Setzen der Zielposition
transform.position = root.position + Vector3.forward * inputVert + Vector3.right * inputHor;
\end{lstlisting}
Durch die Nutzung der Weltachsen anstatt der Hüftrotationsachsen (siehe Listing \ref{lst:nutzersteuerung_2}) kann das Problem behoben werden. Die Steuerung ist nicht relativ zur Ausrichtung des Läufers und dadurch am besten für eine Steuerung aus der Top Down Ansicht geeignet.

\subsection{Versuch 3}
Für die Nutzung in First- oder Thirdperson Ansicht soll die Laufrichtung weiterhin relativ zum Läufer bestimmen werden gleichermaßen aber von der wechselhaften Bewegung des Läufers entkoppeln sein.
Das hinzufügen einer separaten Rotationskomponente für die Bestimmung der Blickrichtung erfüllt hier die Kriterien.
Zu beginn wird die Blickrichtung mit der Vorwärtskomponente der Hüftrotation gleichgesetzt. Ausgehend von der Startrichtung wird dann über horizontalen Mausinput die Richtung angepasst (siehe Listing \ref{lst:nutzersteuerung_3}).
\begin{lstlisting}[caption={Erweiterung der Nutzersteuerung mit separater Blickrichtung},captionpos=b,label={lst:nutzersteuerung_3}]
void Start()
{
    //Root Position als Startposition festhalten
    startForward = root.forward;
    startRight = root.right;
}
void FixedUpdate()
{
    //Einlesen Tastatur Input
    float inputHor = Input.GetAxis("Horizontal");
    float inputVert = Input.GetAxis("Vertical");

    //Einlesen Maus Input
    float mouseX = Input.GetAxis("Mouse X");
    rotAngle += mouseX;

    //Berechnung der Rotation
    Quaternion rotation = Quaternion.AngleAxis(rotAngle, rotationAxis);

    //Anwendung der Rotation auf Richtungsvektoren
    Vector3 directionForward = rotation * startForward;
    Vector3 directionRight = rotation * startRight;

    //Setzen der Zielposition
    transform.position = root.position + directionForward * inputVert + directionRight * inputHor;
}
\end{lstlisting}
Das Ergebnis ermöglicht die Steuerung des Läufers als Spielcharakter, mit einer gewohnten Steuerung aus schon bestehenden Spieltiteln und ist kompatibel mit einer dritte Person Ansicht als auch mit der erste Person Ansicht.
\section{Modell Anpassungen}
Das trainierte Modell der Walker Demo beherrscht jedoch nur die Fortbewegung in Blickrichtung. Der Läufer ist auch nicht darauf trainiert stehen zu bleiben. Das resultiert darin das der Läufer fällt sobald der Nutzer keinen Tastaturinput gibt. Dieses Kapitel beschäftigt sich mit den Einschränkungen der Walker-Demonstration im Bezug auf unterschiedliche Bewegungsrichtungen.

Im ersten Schritt wird getestet wie der Walker angepasst werden kann um die fehlenden Bewegungsabläufe in separaten Modellen zu erlernen.

\subsection{Versuch 4}
Versuch 4 behandelt die Bewegung auf der Stelle stehen. Für das stehenbleiben wird die Zielgeschwindigkeit auf 0 gesetzt während das Ziel auf der Startposition befindet. Die Belohnungsfunktion der Demo, wird ab jetzt Demo Belohnungsfunktion genannt. Die Demo Belohnungsfunktion hat das Problem das durch die Zielgeschwindigkeit geteilt wird, was bei einer Zielgeschwindigkeit von 0 zu Mathematischen Fehlern führt. Um das zu vermeiden wurde das trainieren mit einer anderen Belohnungsfunktion getestet.
\begin{figure}[H]
  \centering  
  \includegraphics[scale=0.5]{img/match_velocity_exp.png}
  \caption{DeepMimic Match Velocity Belohnungsfunktion}
  \label{fig:match_velocity_exp}
\end{figure}
Die neue Belohnungsfunktion ist inspiriert von den Belohnungsfunktionen des Papers "DeepMimic: Example-Guided Deep Reinforcement Learning of Physics-Based Character Skills".\cite{peng2018deepmimic}
Die Belohnungsfunktion aus Abbildung \ref{fig:match_velocity_exp} wird daher ab hier DeepMimic Belohnungsfunktion genannt.

Der Walker konnte mit der DeepMimic Belohnungsfunktion lernen auf der Stelle zu stehen. Mit zufälliger Zielgeschwindigkeit zu einem Ziel zu laufen wie im Ursprünglichen Verhalten konnte damit jedoch nicht zufriedenstellend erlernt werden.

\begin{figure}[H]
  \centering  
  \includegraphics[scale=0.5]{img/versuch4_training}
  \caption{Vergleich von Training mit Demo Belohnungsfunktion gegen DeepMimic Belohnungsfunktion}
  \label{fig:versuch4_training}
\end{figure}

Die Abbildung \ref{fig:versuch4_training} zeigt mit der orangenen Linie die Leistung der DeepMimic Belohnungsfunktion und mit der pinken Linie die Leistung der Demo Belohnungsfunktion. Nachfolgender Vergleich der Belohnungsfunktionen zeigt das die Ursprüngliche Belohnungsfunktion durch das Teilen mit der Zielgeschwindigkeit die Sensitivität der Funktion je nach Zielgeschwindigkeit beeinflusst. Daraus folgt das bei steigender Zielgeschwindigkeit eine größere Abweichung der Geschwindigkeit geduldet wird (siehe Abbildung \ref{fig:match_velocity_demo_vergleich}). Diese Anpassung verbessert die Generalisierung zwischen den wechselnden Geschwindigkeiten um ein vielfaches.

\begin{figure}[H]
  \centering  
  \includegraphics[scale=0.5]{img/match_velocity_demo_vergleich.png}
  \caption{Vergleich der Demo Belohnungsfunktion unter verschiedenen Zielgeschwindigkeiten}
  \label{fig:match_velocity_demo_vergleich}
\end{figure}

\subsection{Versuch 5}
Mit dieser Erkenntnis wurde eine neue Anpassung untersucht. In der folgenden Anpassung blieb die Belohnungsfunktion weitestgehend Unverändert. Lediglich das obere Limit ab welchem die Funktion eine Belohnung von 0 annimmt, wurde auf ein minimum von 0.1 beschränkt. Somit konnte sicher gestellt werden das im Bereich der normalen Fortbewegung keine Veränderung auftritt. Mit der Demo Belohnungsfunktion konnten nur Annäherungen an eine Zielgeschwindigkeit von 0 genutzt werden. Bei einer Annäherung von 0.000001 ist das Spektrum an akzeptablen Geschwindigkeiten bevor die Belohnung 0 ist nahezu unerreichbar (siehe Abbildung \ref{fig:match_velocity_vergleich_clip}). Mit dem Limit von 0.1 ist der Bereich der Belohnungsfunktion > 0 groß genug, sodass der Läufer durch ausprobieren Belohnungen über 0 erreichen kann. Somit kann der Läufer die Belohnung optimieren.\\

\begin{figure}[H]
  \centering  
  \includegraphics[scale=0.5]{img/match_velocity_vergleich_clip.png}
  \caption{Vergleich Demo gegen Belohnungsfunktion mit 0.1 Limit}
  \label{fig:match_velocity_vergleich_clip}
\end{figure}
$V_\delta=Clip(|\vec{Geschwindigkeit} - \vec{Zielgeschwindigkeit}|, 0, |\vec{Zielgeschwindigkeit}|)$ \\
$V_\delta=Clip(|\vec{Geschwindigkeit} - \vec{Zielgeschwindigkeit}|, 0, max(0.1, |\vec{Zielgeschwindigkeit}|))$ \\

Das auf einer Stelle stehen hat der Läufer damit in einem separaten Training auch erlernt. Abbildung \ref{fig:versuch5_training} zeigt das der Läufer die maximale Episoden Länge von 1000 erreicht hat ohne zu fallen. Die bewegte Distanz hat sich auch 0 angenähert.

\begin{figure}[H]
  \centering  
  \includegraphics[scale=0.5]{img/versuch5_training.png}
  \caption{Versuch 5 Traininggraphen}
  \label{fig:versuch5_training}
\end{figure}

Die Belohnungen wurden auch weitestgehend optimiert siehe Abbildung \ref{fig:versuch5_training_belohnung}.

\begin{figure}[H]
  \centering  
  \includegraphics[scale=0.5]{img/versuch5_training_belohnung.png}
  \caption{Versuch 5 Training Belohnungsgraphen}
  \label{fig:versuch5_training_belohnung}
\end{figure}

\subsection{Versuch 6}
Der folgende Versuch untersucht das Laufen in unterschiedliche Richtungen relativ zur Blickrichtung. Um das zu realisieren wurde dem Agent ein enum mit der Laufrichtung hinzugefügt. Die Blickrichtung Belohnungsfunktion wird relativ zur Zielrichtung berechnet. Bei Vorwärtsbewegung ist die Blickrichtung gerade aus. Bei Seitlicher Bewegung ist die Blickrichtung gespiegelt zur Laufrichtung. Läuft der Läufer seitlich rechts ist die Blickrichtung links zum ziel und anders herum. Beim rückwärts gehen ist die Blickrichtung entgegen der Zielrichtung. Die Implementierung ist in \ref{lst:laufrichtung} zu sehen.

\begin{lstlisting}[caption={Laufrichtung Enum, Beobachtung und Belohnung},captionpos=b,label={lst:laufrichtung}]
public enum Direction
{
    Forward,
    Right,
    Left,
    Backward,
}
    
public override void FixedUpdate()
{
    ...
    var headForward = head.forward;
    headForward.y = 0;
    Vector3 lookDirection = cubeForward;
    switch (direction)
    {
        case Direction.Right:
            lookDirection = -walkOrientationCube.transform.right;
            break;
        case Direction.Left:
            lookDirection = walkOrientationCube.transform.right;
            break;
        case Direction.Backward:
            lookDirection = -walkOrientationCube.transform.forward;
            break;
    }
    ...
}
\end{lstlisting}

Das gehen in Zielrichtung wurde durch die Änderungen nicht beeinflusst. Separate Trainings zu den drei anderen Laufrichtungen waren erfolgreich.

\begin{figure}[H]
  \centering  
  \includegraphics[scale=0.5]{img/versuch6_training.png}
  \caption{Versuch 6 Traininggraphen}
  \label{fig:versuch6_training}
\end{figure}

Abbildung \ref{fig:versuch6_training} zeigt die zurückgelegte Distanz und die Anzahl an erreichten Zielen in einer Trainingsepisode. Die Ergebnisse der 3 Laufrichtungen (rot = Rückwärts, gelb = rechts, blau = links) sind alle vergleichbar mit den Ergebnissen der Demo. Die Abweichung der Laufrichtung Links ist vermutlich der zufällligen Natur des Trainings anzurechnen.

\subsection{Versuch 7}
\section{Mixamo Charakter}
-Konfiguration der Physikkomonenten für mixamo Charakter
-Codeanpassung der Konfiguration für mehrere Körperteile
-Vereinfachung durch versteifen von einigen extra Gelenken
-Gallopiert
-Beinwechsel Belohnung um galoppieren zu vermeiden -> funktioniert
-Leistungsminimierung Belohnung um laufverhalten natürlicher zu machen -> funktioniert nur arme sind sehr nah und starr am Körper
-
\section{Gangbild anpassungen}
Das Gangbild des Läufers in der Walker Demo ist sofort als Laufen zu erkennen. Bei genauem hinschauen wird jedoch schnell klar das die Bewegung nicht \hl{natürlich} ist. Das galoppieren des Mixamo Charakters ist auf jedenfall nicht ausreichen um als Laufbewegung durchzugehen.

\subsection{Belohnung für Beinwechsel}
Um sicher zu stellen das der Läufer während dem Training eine Laufbewegung lernt um das Ziel zu erreichen, wird im folgenden Versuch eine Bestrafung eingeführt welche den Läufer bestraft wenn ein Bein zu lange voraus geht.
\begin{figure}[H]
  \centering
  \includegraphics[width=0.9\textwidth]{img/plot_beinwechsel} 
  \caption{Beinwechsel Belohnung}
  \label{fig:plot_beinwechsel}
\end{figure}


\begin{figure}[H]
  \centering
  \begin{tabular}{ccc}
    \includegraphics[width=0.27\textwidth]{img/charakter_mixamo_laufen1} & \includegraphics[width=0.27\textwidth]{img/charakter_mixamo_laufen2}  & \includegraphics[width=0.27\textwidth]{img/charakter_mixamo_laufen3} \\
    \includegraphics[width=0.27\textwidth]{img/charakter_mixamo_laufen4}  & \includegraphics[width=0.27\textwidth]{img/charakter_mixamo_laufen5}  & \includegraphics[width=0.27\textwidth]{img/charakter_mixamo_laufen6} \\
  \end{tabular}
  \caption{Mixamo Versuch 11 Gangbild}
  \label{fig:mixamo_versuch11_gangbild}
\end{figure}

\subsection{Belohnung für Energieminimierung}
Um das Gangbild weiter zu verbessern wird eine Belohnung eingeführt welche den Agenten belohnt wenn er so wenig wie möglich Kraft aufwendet um das Ziel zu erreichen. Genauer gesagt wird er dafür bestraft wenn die Gelenksteuerung einen zu hohe Energiekonsum aufweist.
\begin{figure}[H]
  \centering
  \includegraphics[width=0.9\textwidth]{img/plot_energiespar} 
  \caption{Energiespar Belohnung}
  \label{fig:plot_energiespar}
\end{figure}

\begin{figure}[H]
  \centering
  \begin{tabular}{ccc}
    \includegraphics[width=0.27\textwidth]{img/charakter_mixamo_laufen_energiespar1} & \includegraphics[width=0.27\textwidth]{img/charakter_mixamo_laufen_energiespar2}  & \includegraphics[width=0.27\textwidth]{img/charakter_mixamo_laufen_energiespar3} \\
    \includegraphics[width=0.27\textwidth]{img/charakter_mixamo_laufen_energiespar4}  & \includegraphics[width=0.27\textwidth]{img/charakter_mixamo_laufen_energiespar5}  & \includegraphics[width=0.27\textwidth]{img/charakter_mixamo_laufen_energiespar6} \\
  \end{tabular}
  \caption{Mixamo Versuch 12 Gangbild}
  \label{fig:mixamo_versuch12_gangbild}
\end{figure}

\subsection{Imitationslernen}

-Beinwechsel Belohnung um galoppieren zu vermeiden -> funktioniert
-Leistungsminimierung Belohnung um laufverhalten natürlicher zu machen -> funktioniert nur arme sind sehr nah und starr am Körper
-

\input{06_fazit}

\appendix

%Literaturverzeichnis
\printbibliography[title=Literaturverzeichnis]

\end{document}