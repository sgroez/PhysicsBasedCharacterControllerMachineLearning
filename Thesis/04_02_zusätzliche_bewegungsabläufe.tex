\section{Zusätzliche Bewegungsabläufe}
\subsection{Anforderungen}
Dieses Kapitel beschäftigt sich mit den Einschränkungen der Walker-Demonstration im Bezug auf unterschiedliche Bewegungsabläufe. Das trainierte Modell der Walker-Demo beherrscht derzeit nur die Fortbewegung in Blickrichtung und ist nicht in der Lage, auf der Stelle stehen zu bleiben. Dies führt dazu, dass der Läufer stürzt, sobald der Nutzer keinen Tastaturinput mehr gibt. Diese Einschränkungen lassen sich auf die Art und Weise zurückführen, wie das Modell trainiert wurde, da es ausschließlich auf die Vorwärtsbewegung optimiert wird. Um diese Probleme zu beheben, werden Anpassungen am Trainingsablauf vorgenommen. Insbesondere wird untersucht, wie das Modell so erweitert werden kann, dass es unabhängig von der Bewegungsrichtung eine stabile Blickrichtung beibehält und die Fähigkeit erlangt, auf der Stelle zu stehen.

\subsection{Separate Bewegungsabläufe}

Das erste Ziel ist es, dem Läufer das Stehenbleiben beizubringen. Hierfür soll der Läufer sich möglichst wenig von der aktuellen Position wegbewegen, wobei die Hauptaufgabe darin besteht, das Gleichgewicht zu halten.

Um das Stehenbleiben zu erreichen, wird die Zielgeschwindigkeit auf 0 gesetzt, während das Ziel an der Startposition bleibt. Eine Herausforderung hierbei ist die ursprüngliche Demo-Belohnungsfunktion, die durch die Zielgeschwindigkeit dividiert. Da dies bei einer Zielgeschwindigkeit von 0 zu mathematischen Fehlern führt, war eine Anpassung der Belohnungsfunktion notwendig. Statt der ursprünglichen Funktion wurde eine alternative Belohnungsfunktion implementiert, um das Problem zu beheben. Es wird erwartet, dass die Neue Belohnungsfunktion die Lernfähigkeit des Modells verbessert, auf der Stelle zu stehen, ohne zu fallen. Es muss jedoch auch untersucht werden wie die Neue Belohnungsfunktion die Lernfähigkeit des ursprünglichen Verhaltens beeinflusst.

\begin{figure}[H]
  \centering  
  \includegraphics[width=0.9\textwidth]{img/plot_vel_reward_neu}
  \caption{Neue Sigmoid Geschwindigkeitbelohnungsfunktion}
  \label{fig:plot_vel_reward_neu}
\end{figure}

Als Neue Belohnungsfunktion wird eine Sigmoid-Funktion genutzt, die eine glatte Abstufung der Belohnungen ermöglicht. Die Belohnung erreicht den Wert 1, wenn die aktuelle Geschwindigkeit perfekt mit der Zielgeschwindigkeit übereinstimmt. Mit zunehmender Abweichung von der Zielgeschwindigkeit sinkt die Belohnung stetig, und sobald die Abweichung den Wert 1 überschreitet, fällt die Belohnung auf 0. Die Belohnungsfunktion und ihre Parameter wurden so gewählt, dass die Neue Belohnungsfunktion der Demo Belohnungsfunktion ähnelt. Die Neue Belohnungsfunktion abgebildet in Abbildung \ref{fig:plot_vel_reward_neu} wird daher ab hier Neue Belohnungsfunktion genannt.

Der Walker konnte mit der Neuen Belohnungsfunktion erfolgreich lernen, auf der Stelle zu stehen. Dies wird in den Abbildungen \ref{fig:126_episode_length} und \ref{fig:126_move_target_dir} verdeutlicht. Die Abbildung \ref{fig:126_move_target_dir} zeigt, wie die zurückgelegte Distanz um 0 herum schwankt, was darauf hinweist, dass der Läufer in der Lage ist, seine Position zu halten. Gleichzeitig erreicht die Episodenlänge in Abbildung \ref{fig:126_episode_length} die maximale Länge von 1000, was bedeutet, dass der Läufer über die gesamte Episode hinweg stabil blieb.
\begin{figure}[H]
  \centering  
  \begin{subfigure}{.49\textwidth}
      \centering  
      \includegraphics[width=\textwidth]{img/126_episode_length}
      \caption{Episodenlänge}
      \label{fig:126_episode_length}
    \end{subfigure}
    \begin{subfigure}{.49\textwidth}
      \centering  
      \includegraphics[width=\textwidth]{img/126_move_target_dir}
      \caption{Zurückgelegte Stecke in Zielrichtung}
      \label{fig:126_move_target_dir}
    \end{subfigure}
  \caption{Training stehen mit neuer Belohnungsfunktion}
  \label{fig:training_stehen_neu}
\end{figure}

Mit zufälliger Zielgeschwindigkeit zu einem Ziel zu laufen, wie es im ursprünglichen Verhalten vorgesehen war, konnte jedoch mit der Neuen Belohnungsfunktion nicht zufriedenstellend erlernt werden. In Abbildung \ref{fig:training_vergleich_demo_neu} ist die Leistung der Neuen Belohnungsfunktion durch die orangene Linie und die Leistung der Demo-Belohnungsfunktion durch die rosa Linie dargestellt. Die Abbildung \ref{fig:match_velocity_demo_vergleich} zeigt, dass die ursprüngliche Belohnungsfunktion die Fehlertoleranz in Abhängigkeit von der Zielgeschwindigkeit dynamisch anpasst, was dem Modell ermöglicht, besser zwischen unterschiedlichen Geschwindigkeiten zu generalisieren. Die Ergebnisse zeigen, dass während die Neue Belohnungsfunktion gut für das Erlernen der Stabilisierung des Läufers auf der Stelle ist, die ursprüngliche Belohnungsfunktion sich besser für das Erlernen einer Gangbewegung in Zielrichtung eignet.

\begin{figure}[H]
  \centering  
  \begin{subfigure}{.49\textwidth}
      \centering  
      \includegraphics[width=\textwidth]{img/116_127_move_target_dir}
      \caption{Zurückgelegte Stecke in Zielrichtung}
      \label{fig:116_127_move_target_dir}
    \end{subfigure}
    \begin{subfigure}{.49\textwidth}
      \centering  
      \includegraphics[width=\textwidth]{img/116_127_reach_target}
      \caption{Anzahl erreichte Ziele}
      \label{fig:116_127_reach_target}
    \end{subfigure}
  \caption{Vergleich von Lauftraining mit Demo Belohnungsfunktion gegen Neue Belohnungsfunktion}
  \label{fig:training_vergleich_demo_neu}
\end{figure}

\begin{figure}[H]
  \centering  
  \includegraphics[width=0.9\textwidth]{img/match_velocity_demo_vergleich}
  \caption{Vergleich der Demo Belohnungsfunktion unter verschiedenen Zielgeschwindigkeiten}
  \label{fig:match_velocity_demo_vergleich}
\end{figure}

Mit dieser Erkenntnis wird eine neue Anpassung untersucht. In der folgenden Anpassung bleibt die Belohnungsfunktion weitestgehend unverändert; lediglich das obere Limit, ab welchem die Funktion eine Belohnung von 0 annimmt, wird auf ein Minimum von 0,1 beschränkt. Diese Anpassung stellt sicher, dass im Bereich der normalen Fortbewegung keine unerwünschten Veränderungen auftreten. Das Problem mit der ursprünglichen Demo-Belohnungsfunktion bestand darin, dass bei Annäherung an eine Zielgeschwindigkeit von 0 das Spektrum an akzeptablen Geschwindigkeiten, bevor die Belohnung auf 0 sinkt, extrem eng wurde (siehe Abbildung \ref{fig:match_velocity_vergleich_clip}). Dies machte es nahezu unmöglich, sinnvolle Lernfortschritte in diesem Bereich zu erzielen. Durch die Einführung eines Limits von 0,1 wird der Bereich der Geschwindigkeitsabweichungen, für die die Belohnungsfunktion einen Wert größer 0 annimmt, ausreichend vergrößert. Dies ermöglicht dem Läufer, durch Ausprobieren Belohnungen über 0 zu erreichen, was wiederum eine Richtung für die Optimierung des Verhaltens bietet. Somit kann der Läufer die Belohnung optimieren.

\begin{figure}[H]
  \centering  
  \includegraphics[width=0.9\textwidth]{img/match_velocity_vergleich_clip}
  \caption{Vergleich Demo gegen Belohnungsfunktion mit 0.1 Limit}
  \label{fig:match_velocity_vergleich_clip}
\end{figure}

Mit der angepassten Demo-Belohnungsfunktion konnte nach etwas mehr Trainingsschritten ebenfalls die maximale Episodenlänge von 1000 erreicht werden (siehe Abbildung \ref{fig:126_128_episode_length}. Dies zeigt, dass der Läufer in der Lage war, über eine längere Trainingsdauer hinweg die erforderliche Stabilität zu erlernen. Die in Abbildung \ref{fig:126_128_move_target_dir} dargestellte zurückgelegte Distanz nähert sich gegen Ende fast 0, was darauf hindeutet, dass der Läufer seine Position sehr gut halten kann. Dieses Ergebnis zeigt, dass auch mit der angepassten Demo-Belohnungsfunktion ein vergleichbares Niveau an Stabilität erreicht werden kann, ohne den Lernvorgang für das ursprüngliche Verhalten zu beeinflussen.

\begin{figure}[H]
  \centering  
  \begin{subfigure}{.49\textwidth}
      \centering  
      \includegraphics[width=\textwidth]{img/126_128_episode_length}
      \caption{Episodenlänge}
      \label{fig:126_128_episode_length}
    \end{subfigure}
    \begin{subfigure}{.49\textwidth}
      \centering  
      \includegraphics[width=\textwidth]{img/126_128_move_target_dir}
      \caption{Zurückgelegte Stecke in Zielrichtung}
      \label{fig:126_128_move_target_dir}
    \end{subfigure}
  \caption{Vergleich Training stehen mit Neuer Belohnungfunktion (orange) und angepasster Demo Belohnungsfunktion (blau)}
  \label{fig:vergleich_126_128}
\end{figure}

Nach dem erfolgreichen Erlernen des Stehens auf einer festen Position sind die nächsten Bewegungsziele die Fortbewegung zum Ziel mit unterschiedlichen Blickrichtungen. Die Extremfälle umfassen dabei das Rückwärts- und Seitwärtslaufen. In diesem Abschnitt wird untersucht, ob der Läufer in der Lage ist, Bewegungen in verschiedene Richtungen relativ zu seiner Blickrichtung zu erlernen. Um dies zu ermöglichen, wurde die Belohnungsfunktion angepasst, sodass sie die Blickrichtung relativ zur Zielrichtung berücksichtigt. Bei der Vorwärtsbewegung entspricht die Blickrichtung der Zielrichtung. Bei der Seitwärtsbewegung steht die Blickrichtung im rechten Winkel zur Zielrichtung, und bei der Rückwärtsbewegung ist die Blickrichtung entgegengesetzt zur Zielrichtung. Die Implementierung zur Bestimmung der Blickrichtung ist in \ref{lst:blickrichtung} dargestellt.

\begin{lstlisting}[caption={Blickrichtung festlegen mit Richtungs Enum},captionpos=b,label={lst:blickrichtung}]
public enum Direction
{
    Forward,
    Right,
    Left,
    Backward,
}
    
public override void FixedUpdate()
{
    ...
    var headForward = head.forward;
    headForward.y = 0;
    Vector3 lookDirection = cubeForward;
    switch (direction)
    {
        case Direction.Right:
            lookDirection = -walkOrientationCube.transform.right;
            break;
        case Direction.Left:
            lookDirection = walkOrientationCube.transform.right;
            break;
        case Direction.Backward:
            lookDirection = -walkOrientationCube.transform.forward;
            break;
    }
    var lookAtTargetReward = (Vector3.Dot(lookDirection, headForward) + 1) * 0.5F;
    ...
}
\end{lstlisting}

Das Gehen in Zielrichtung wurde durch die Änderungen nicht negativ beeinflusst. Separate Trainings für die drei anderen Laufrichtungen – seitlich nach links, seitlich nach rechts und rückwärts – waren ebenfalls erfolgreich. Abbildungen \ref{fig:116_130_131_132_move_target_dir} und \ref{fig:116_130_131_132_reach_target} zeigen die zurückgelegte Distanz und die Anzahl der erreichten Ziele in einer Trainingsepisode. Die Ergebnisse für die drei Laufrichtungen sind alle vergleichbar mit den Ergebnissen der ursprünglichen Demo. Die Tatsache, dass die Ergebnisse  \grqq{}vergleichbar \grqq{} sind, bedeutet, dass der Läufer in der Lage war, eine ähnliche Anzahl von Zielen zu erreichen und ähnliche Distanzen zurückzulegen wie in der ursprünglichen Demo, unabhängig von der Laufrichtung. Insgesamt zeigen die Ergebnisse, dass der Läufer mit den Einschränkungen der Gelenke und der implementierten Belohnungen dazu in der Lage ist, die Bewegung zum Ziel mit allen vier Blickrichtungen zu meistern. Bei der Bewegungsrichtung seitlich - links ist jedoch eine starke Abweichung zu verzeichnen. Diese Abweichung zeigt sich in einer geringeren Übereinstimmung der Geschwindigkeit sowie weniger erreichten Zielen und einer kürzeren zurückgelegten Distanz im Vergleich zur seitlichen Bewegung nach rechts. Da die seitlichen Bewegungsrichtungen symmetrisch zueinander identische Ergebnisse erzielen sollten, wird davon ausgegangen, dass diese Abweichung auf die zufällige Natur des maschinellen Lernprozesses und der Trainingsumgebung zurückzuführen ist. Die zufällige Platzierung des Laufziels kann beispielsweise durch das platzieren von weiter entfernten Zielen die Schwierigkeit zufällig beeinflussen. Weiterhin kann auch die zufällig gewählte Geschwindigkeit und Startrotation die Schwierigkeit verändern. Generell sollten sich die Abweichungen über die vielen Trainingsepisoden relativieren, es ist jedoch nicht auszuschließen, dass das Training dadurch unterschiedlich verläuft.

\begin{figure}[H]
  \centering  
  \begin{subfigure}{.49\textwidth}
      \centering  
      \includegraphics[width=\textwidth]{img/116_130_131_132_episode_length}
      \caption{Episodenlänge}
      \label{fig:116_130_131_132_episode_length}
    \end{subfigure}
    \begin{subfigure}{.49\textwidth}
      \centering  
      \includegraphics[width=\textwidth]{img/116_130_131_132_cumulative_reward}
      \caption{Angehäufte Belohnung}
      \label{fig:116_130_131_132_cumulative_reward}
    \end{subfigure}
     \begin{subfigure}{.49\textwidth}
      \centering  
      \includegraphics[width=\textwidth]{img/116_130_131_132_look_reward}
      \caption{Blickbelohnung}
      \label{fig:116_130_131_132_look_reward}
    \end{subfigure}
    \begin{subfigure}{.49\textwidth}
      \centering  
      \includegraphics[width=\textwidth]{img/116_130_131_132_vel_reward}
      \caption{Geschwindigkeitsbelohnung}
      \label{fig:116_130_131_132_vel_reward}
    \end{subfigure}
    \begin{subfigure}{.49\textwidth}
      \centering  
      \includegraphics[width=\textwidth]{img/116_130_131_132_move_target_dir}
      \caption{Zurückgelegte Stecke in Zielrichtung}
      \label{fig:116_130_131_132_move_target_dir}
    \end{subfigure}
    \begin{subfigure}{.49\textwidth}
      \centering  
      \includegraphics[width=\textwidth]{img/116_130_131_132_reach_target}
      \caption{Anzahl erreichte Ziele}
      \label{fig:116_130_131_132_reach_target}
    \end{subfigure}
  \caption{Unterschiedliche Blickrichtungen Training Graphen (grün = vorwärts, orange = rückwärts, rosa = rechts, blau = links)}
  \label{fig:training_unterschiedliche_blickrichtung}
\end{figure}
\subsection{Laufrichtungen kombinieren}
Die Charaktersteuerung benötigt je nach Tastatureingabe eine unterschiedliche Bewegungsrichtung. Der Unity ML-Agents Agent enthält eine Funktion zum wechseln des verwendeten Modells. Mit dieser Funktion wird in folgender Implementierung zwischen den seperaten Bewegungsmodellen gewechselt um alle Bewegungsrichtungen mit einer Steuerung abzudecken. Zu Erwarten ist das die Bewegung in die einzelnen Richtungen funktioniert, der Läufer aber beim Wechsel zwischen den Modellen das Gleichgewicht nicht halten kann.

\begin{lstlisting}[caption={Laufrichtung Modell wechseln},captionpos=b,label={lst:laufrichtung_modell_wechsel}]
public override void FixedUpdate() {
    ...    
    agent.targetWalkingSpeed = 5f;
    if (inputVert != 0) //Tastatur Input Vor oder Zurück
    {
        // Vorwärts
        if (inputVert > 0)
        {
            agent.SetModel("Walker", modelForward);
        }
        else // Zurück
        {
            agent.SetModel("Walker", modelBackward);
        }
    }
    else if (inputHor != 0) // Links oder Rechts
    {
        if (inputHor > 0) // Rechts
        {
            agent.SetModel("Walker", modelRight);
        }
        else // Links
        {
            agent.SetModel("Walker", modelLeft);
        }
    }
    else //kein Input -> Auf der Stelle stehen
    {
        agent.targetWalkingSpeed = 0f;
        agent.SetModel("Walker", modelStanding);
    }
    ...
}
\end{lstlisting}

Wie angenommen funktioniert das Bewegen in eine konstante Richtung gut. Beim Wechsel zu einem anderen Modell fällt der Läufer ohne Ausnahme.

Der erste Versuch alle Bewegungsrichtungen in einem Modell anzulernen, wurde von den Methoden der Walker Demo inspiriert. Gleich wie das Laufziel wurde ein zusätzliches Zielobjekt hinzugefügt, welches zufällig platziert wurde. Um die Komplexität nicht von Anfang an zu hoch anzusetzen, wurde das Blickziel zu beginn die Winkelabweichung zwischen Zielrichtung und Blickrichtung langsam über ein Lehrplan angepasst. Anfangs wurde das Blickziel mit einer Winkelabweichung im Bereich von -5 und 5 Grad platziert, zum Ende hin wurde der Bereich immer weiter bis auf -90 bis 90 Grad erweitert (siehe Codeausschnitt \ref{lst:lehrplan_blickziel}). Das Blickziel wurde neu gesetzt sobald ein Ziel erreicht wurde.

\begin{lstlisting}[caption={ Lehrplan für das Blickziel},captionpos=b,label={lst:lehrplan_blickziel}]
lookAngleLimit:
    curriculum:
      - name: min max 5 degree look target deviation
        completion_criteria:
          measure: progress
          behavior: Walker
          threshold: 0.4
          signal_smoothing: true
        value: 5.0
      - name: min max 30 degree look target deviation
        completion_criteria:
          measure: progress
          behavior: Walker
          threshold: 0.6
          signal_smoothing: true
          require_reset: true
        value: 30.0
      - name: min max 60 degree look target deviation
        completion_criteria:
          measure: progress
          behavior: Walker
          threshold: 0.8
          signal_smoothing: true
          require_reset: true
        value: 60.0
      - name: min max 90 degree look target deviation
        completion_criteria:
          measure: progress
          behavior: Walker
          threshold: 1.0
          signal_smoothing: true
          require_reset: true
        value: 90.0
\end{lstlisting}

Der Läufer lernt einen stabilen Gang (siehe Abbildungen \ref{fig:103_move_target_dir} und \ref{fig:103_reach_target}). Die Winkelabweichung von 5 Grad ist mit aktueller Belohnungsfunktion jedoch nahezu zu vernachlässigen. Vor folgenden Lehreinheiten hat sich bereits ein Verhalten so start gefestigt, das die Lehreinheiten keine Wirkung zeigen (siehe Abbildungen \ref{fig:103_look_angle_limit} und \ref{fig:126_look_reward}.

\begin{figure}[H]
  \centering  
    \begin{subfigure}{.49\textwidth}
      \centering  
      \includegraphics[width=\textwidth]{img/103_move_target_dir}
      \caption{Zurückgelegte Strecke in Zielrichtung}
      \label{fig:103_move_target_dir}
    \end{subfigure}
    \begin{subfigure}{.49\textwidth}
      \centering  
      \includegraphics[width=\textwidth]{img/103_reach_target}
      \caption{Erreichte Anzahl an Zielen}
      \label{fig:103_reach_target}
    \end{subfigure}
    \begin{subfigure}{.49\textwidth}
      \centering  
      \includegraphics[width=\textwidth]{img/103_look_angle_limit}
      \caption{Aktive Lehreinheit}
      \label{fig:103_look_angle_limit}
    \end{subfigure}
     \begin{subfigure}{.49\textwidth}
      \centering  
      \includegraphics[width=\textwidth]{img/103_look_reward}
      \caption{Blickbelohnung}
      \label{fig:103_look_reward}
    \end{subfigure}
  \caption{Training Blickrichtungsziel mit Lehrplan}
  \label{fig:training_blickrichtungsziel_lehrplan}
\end{figure}

Um von Beginn an ein generell gültiges Verhalten anzutrainieren, wurden Trainings ohne Lehrplan durchgeführt. Dabei wurde ein Training mit einer Winkelabweichung von +-90 Grad und eins mit +-180 Grad durchgeführt. Der Läufer lernt bei beiden Trainings kaum die Blickrichtung der neuen Zielblickrichtung anzupassen. Bei Training mit Winkelabweichungen von bis zu +-180 Grad lernt der Läufer negative Belohnungen mit einer Blickrichtung vertikal nach unten zu umgehen. Dies ist möglich da die die Blickrichtung Vertikal nach unten entlang der Y-Achse verläuft. Bei der Berechnung der Blickbelohnung wird die Y-Komponente der Blickrichtung auf 0 gesetzt, da die Zielrichtung auch ohne Höhenkomponente festgelegt wird um Höhenunterschiede zwischen Hüfte und Ziel zu ignorieren. Durch dieses Schlupfloch kann der Läufer negative Belohnungen verhindern ohne die Gangart nennenswert anzupassen.

\begin{figure}[H]
  \centering  
    \begin{subfigure}{.49\textwidth}
      \centering  
      \includegraphics[width=\textwidth]{img/104_105_move_target_dir}
      \caption{Zurückgelegte Strecke in Zielrichtung}
      \label{fig:104_105_move_target_dir}
    \end{subfigure}
    \begin{subfigure}{.49\textwidth}
      \centering  
      \includegraphics[width=\textwidth]{img/104_105_reach_target}
      \caption{Erreichte Anzahl an Zielen}
      \label{fig:104_105_reach_target}
    \end{subfigure}
    \begin{subfigure}{.49\textwidth}
      \centering  
      \includegraphics[width=\textwidth]{img/104_105_look_reward}
      \caption{Blickbelohnung}
      \label{fig:104_105_look_reward}
    \end{subfigure}
  \caption{Training Blickrichtungsziel (blau = Winkelabweichung +-90 Grad, grün = Winkelabweichung +-180 Grad)}
  \label{fig:training_blickrichtungsziel}
\end{figure}

Um das Schlupfloch der Blickbelohnung zu stopfen wird eine neue Bestrafung eingeführt. Die Kopfneigungsbestrafung bestraft den Läufer für das neigen des Kopfs und sorgt somit dafür das der Läufer den Kopf aufrecht hält. Im Training mit der neuen Kopfneigungsbestrafung lernt der Läufer langsamer und findet eine neuen Ausweg die Belohnungen generell gültig zu optimieren ohne unterschiedliche Gangarten zu erlernen. Durch die Eigenschaft des Trainings, dass der Winkel zwischen Läufer, Ziel und Blickziel nur bei der Platzierung des Blickziels eingehalten werden, werden die Winkel durch das annähern des Läufers zum Ziel ausnahmslos größer. Somit ist die beste Lösung für den Läufer sich rückwärts zu bewegen, denn so ist die Wahrscheinlichkeit das die Winkelabweichung kleiner ist wesentlich höher (siehe Abbildung \ref{fig:blickwinkel_änderung}).

\begin{figure}[H]
  \centering  
    \begin{subfigure}{.3\textwidth}
      \centering  
      \includegraphics[width=\textwidth]{img/blickwinkel_änderung1}
    \end{subfigure}
    \begin{subfigure}{.3\textwidth}
      \centering  
      \includegraphics[width=\textwidth]{img/blickwinkel_änderung2}
    \end{subfigure}
    \begin{subfigure}{.3\textwidth}
      \centering  
      \includegraphics[width=\textwidth]{img/blickwinkel_änderung3}
    \end{subfigure}
  \caption{Blickwinkel Änderung durch Zielannäherung}
  \label{fig:blickwinkel_änderung}
\end{figure}

Das Problem wurde behoben, indem das Blickziel in nachfolgenden Trainings kontinuierlich mit jedem Update neu platziert wurde, um somit den Blickwinkel gleich zu halten. In den bisherigen Trainingseinheiten hat der Läufer zudem eine Gangart optimiert um allen Blickziele mittelmäßig zu erreichen. Um den Läufer zu motivieren alle Blickziele stärker einzuhalten, wurde die Implementierung geändert, sodass der Blickwinkel beim erreichen des Laufziels nur noch gewechselt wird wenn die Durchschnittliche Blickbelohnung über dem Schwellenwert von 0.7 liegt. Mit dieser neuen Bedingung beim erreichen des Blickziels, erreicht der Läufer die ersten Blickziele und stagniert dann an den neuen Blickzielen (siehe \ref{fig:training_blickrichtungsziel_wechsel_07}).

\begin{figure}[H]
  \centering  
    \begin{subfigure}{.49\textwidth}
      \centering  
      \includegraphics[width=\textwidth]{img/113_move_target_dir}
      \caption{Zurückgelegte Strecke in Zielrichtung}
      \label{fig:113_move_target_dir}
    \end{subfigure}
    \begin{subfigure}{.49\textwidth}
      \centering  
      \includegraphics[width=\textwidth]{img/113_reach_target}
      \caption{Erreichte Anzahl an Zielen}
      \label{fig:113_reach_target}
    \end{subfigure}
    \begin{subfigure}{.49\textwidth}
      \centering  
      \includegraphics[width=\textwidth]{img/113_look_reward}
      \caption{Blickbelohnung}
      \label{fig:113_look_reward}
    \end{subfigure}
  \caption{Training Blickrichtungsziel mit Wechsel bei durchschnittlicher Blickbelohnung von 0.7}
  \label{fig:training_blickrichtungsziel_wechsel_07}
\end{figure}

Der Läufer braucht zu beginn eine ganze Weile um zu lernen sich bis zum Ziel zu bewegen. Aus diesem Grund verbringt der Läufer auch mit der neuen Bedingung zum erreichen des Blickziels zu viel Zeil des Trainings mit gleichbleibenden Blickzielwinkeln. Um von Anfang an und separat vom erlernen des Laufens die Blickbelohnung zu erlernen, wird die Bedingung für das Erreichen eines Blickziels erneut angepasst. Im folgenden Versuch wird der Blickzielwinkel gewechselt sobald der Läufer 3 Sekunden auf das Ziel blickt. Implementiert wurde das ganze mit einem Spherecast um gleichzeitig zu ermöglichen die Genauigkeit mit der der Läufer auf das Ziel schauen muss anpassen zu können (siehe Abbildung \ref{fig:spherecast}).

\begin{figure}[H]
  \centering  
  \includegraphics[width=0.8\textwidth]{img/spherecast}
  \caption{Spherecast in Blickrichtung}
  \label{fig:spherecast}
\end{figure}

Das Training wurde einmal mit 3 Sekunden und einmal mit 2 Sekunden Blickkontaktzeit zum erreichen des Blickziels durchgeführt. Wobei das Training mit 2 Sekunden Blickkontaktzeit besser abschneidet. Aber auch das bessere der beiden Trainings ist nach 60 bzw. 120 milionen Trainingsschritten noch weit davon entfernt stabil mehrere Ziele am Stück zu erreichen.

\begin{figure}[H]
  \centering  
    \begin{subfigure}{.49\textwidth}
      \centering  
      \includegraphics[width=\textwidth]{img/117_119_move_target_dir}
      \caption{Zurückgelegte Strecke in Zielrichtung}
      \label{fig:117_119_move_target_dir}
    \end{subfigure}
    \begin{subfigure}{.49\textwidth}
      \centering  
      \includegraphics[width=\textwidth]{img/117_119_reach_target}
      \caption{Erreichte Anzahl an Zielen}
      \label{fig:117_119_reach_target}
    \end{subfigure}
    \begin{subfigure}{.49\textwidth}
      \centering  
      \includegraphics[width=\textwidth]{img/117_119_reach_look_target}
      \caption{Erreichte Anzahl an Blickzielen}
      \label{fig:117_119_reach_look_target}
    \end{subfigure}
    \begin{subfigure}{.49\textwidth}
      \centering  
      \includegraphics[width=\textwidth]{img/117_119_look_reward}
      \caption{Blickbelohnung}
      \label{fig:117_119_look_reward}
    \end{subfigure}
  \caption{Training Blickrichtungsziel mit Wechsel bei Blickkontakt von 2 bzw. 3 Sekunden (orange = 3 sek, grün = 2 sek)}
  \label{fig:training_blickrichtungsziel_wechsel_spherecast}
\end{figure}