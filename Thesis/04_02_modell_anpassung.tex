\section{Modell Anpassungen}
Dieses Kapitel beschäftigt sich mit den Einschränkungen der Walker-Demonstration im Bezug auf unterschiedliche Bewegungsrichtungen.
stehen bleiben:
wenn Distance < slowDownDistance dann je nach Distanz die Geschwindigkeit verringern -> agent bleibt vor ziel stehen bzw. kreist um ziel

extra Laufrichtungen:
-getrennte Modelle und Modell live wechseln problem mit seitwärts laufen und vermutlich schlechter Übergang bei Wechsel
-ein Modell mit Laufrichtung als one hot encoding in Beobachtung mit Lektion für verschiedene Laufrichtung -> kann nicht von vorwärtslaufen auf andere Richtung generalisieren bzw. anpassen (vergisst vorheriges verhalten) schränkt Bewegung auf feste Bewegungsrichtungen ein (vorwärts, rechts, links, rückwärts)
-extra Ziel für Blickrichtung:
  -Ziel zufällig gesetzt mit Winkel Begrenzung von agent zu ziel -> Winkel ändert sich bei Bewegung
  -Blickziel setzen bei Episodenwechsel oder Ziel erreicht -> ändert sich zu häufig das Agent verhalten nicht lernt
  -Blickziel und Laufziel nur neu setzen wenn Durchschnittliche Blickbelohnung > Grenzwert -> zu schwer bzw. dauert zu lange, Agent veralten schon zu sehr vertieft um es groß zu ändern
  -Walker lernt auf Boden zu schauen da Blickrichtung nach unten näher an Blickrichtung Ziel ist wenn sich das Ziel hinter dem Walker befindet
  -Extra Belohnung für aufrechte Blickrichtung
  -Blickziel wird jedes Physikupdate neu gesetzt um Winkel gleich zu behalten
  -Blickziel neu setzen wenn bestimmte Zeit auf Ziel geschaut (mit Spherecast) -> funktioniert nicht schlecht aber bei längerem training hört der Agent auf das Blickziel zu erreichen